% This template was initially provided by Dulip Withanage.
% Modifications for the database systems research group
% were made by Conny Junghans,  Jannik Strötgen and Michael Gertz

\documentclass[
     12pt,         % font size
     a4paper,      % paper format
     BCOR10mm,     % binding correction
     DIV14,        % stripe size for margin calculation
%     liststotoc,   % table listing in toc
%     bibtotoc,     % bibliography in toc
%     idxtotoc,     % index in toc
%     parskip       % paragraph skip instad of paragraph indent
     ]{scrreprt}

%%%%%%%%%%%%%%%%%%%%%%%%%%%%%%%%%%%%%%%%%%%%%%%%%%%%%%%%%%%%

% PACKAGES:

% Use German :
\usepackage[english]{babel}
% Input and font encoding
\usepackage[utf8]{inputenc}
\usepackage[T1]{fontenc}
% Index-generation
\usepackage{makeidx}
% Einbinden von URLs:
\usepackage{url}
% Special \LaTex symbols (e.g. \BibTeX):
%\usepackage{doc}
% Include Graphic-files:
\usepackage{graphicx}
% Include doc++ generated tex-files:
%\usepackage{docxx}
% Include PDF links
%\usepackage[pdftex, bookmarks=true]{hyperref}
% \usepackage[acronym, nonumberlist]{glossaries}

\usepackage[acronym, nonumberlist]{glossaries}
\usepackage{glossaries-extra}

\usepackage{csquotes}

\usepackage{amsfonts}
\usepackage{algorithm}
\usepackage{algorithmicx}
\usepackage{algpseudocode}
\usepackage{amsmath}


\usepackage{listings}
\usepackage{booktabs}
\usepackage{pdflscape}
\usepackage{longtable}
\usepackage{float}
\usepackage{rotating}
\usepackage{subcaption}
% \usepackage{caption}

\usepackage[width=.75\textwidth]{caption}

\usepackage{breakcites}
\usepackage{multirow}
\usepackage{diagbox}





% Fuer anderthalbzeiligen Textsatz
\usepackage{setspace}

% hyperrefs in the documents
\usepackage[bookmarks=true,colorlinks,pdfpagelabels,pdfstartview = FitH,bookmarksopen = true,bookmarksnumbered = true,linkcolor = black,plainpages = false,hypertexnames = false,citecolor = black,urlcolor=black]{hyperref} 
%\usepackage{hyperref}

%%%%%%%%%%%%%%%%%%%%%%%%%%%%%%%%%%%%%%%%%%%%%%%%%%%%%%%%%%%%

% OTHER SETTINGS:

% Pagestyle:
\pagestyle{headings}

% Choose language
\newcommand{\setlang}[1]{\selectlanguage{#1}\nonfrenchspacing}

% Acronym definition
\makeglossaries
\newacronym{qa}{QA}{Question Answering}
\newacronym{ai}{AI}{Artificial Intelligence}
\newacronym{ml}{ML}{Machine Learning}
\newacronym{dl}{DL}{Deep Learning}
\newacronym{convqa}{Conv QA}{Conversational Question Answering}
\newacronym{mit}{MIT}{Massachusetts Institute of Technology}
\newacronym{trec}{TREC}{Text REtrieval Conference}
\newacronym{ir}{IR}{Information Retrieval}
\newacronym{nn}{NN}{Neural Network}
\newacronym{s2s}{seq-2-seq}{Sequence-to-Sequence}
\newacronym{kb}{KB}{Knowledge Base}
\newacronym{mrc}{MRC}{Machine Reading Comprehension}
\newacronym{rag}{RAG}{Retrieval-Augmented Generation}
\newacronym{realm}{REALM}{Retrieval-Augmented Language Model pre-training}
\newacronym{docvqa}{DocVQA}{Document Visual Question Answering}
\newacronym{ocr}{OCR}{Optical Character Recognition}
\newacronym{llm}{LLM}{Large Language Model}
\newacronym{dpr}{DPR}{Dense Passage Retrieval}
\newacronym{bert}{BERT}{Bidirectional Encoder Representations from Transformers}
\newacronym{prlm}{PrLM}{Transformer-based Pre-trained Language Model}
\newacronym{qg}{QG}{Automatic Question Generation}
\newacronym{odqa}{ODQA}{Open-Domain Question Answering}
\newacronym{laprador}{LaPraDoR}{\textbf{La}rge-scale \textbf{Pr}etr\textbf{a}ined \textbf{D}ense Zer\textbf{o}-shot \textbf{R}etriever}
\newacronym{ood}{OOD}{Out-of-Domain}
\newacronym{r2d2}{R2-D2}{\textbf{R}ank Twice, Rea\textbf{d} Twice}
\newacronym{cis}{CIS}{Conversational Information-System}
\newacronym{odcqa}{ODCQA}{Open-Domain Conversational Question Answering}
\newacronym{coqa}{CoQA}{Conversational Question Answering}
\newacronym{quretec}{QuReTeC}{\textbf{Qu}ery \textbf{Re}solution by \textbf{Te}rm \textbf{C}lassification}
\newacronym{llama}{Llama}{\textbf{L}arge \textbf{La}nguage \textbf{M}odels with \textbf{A}dapters}
\newacronym{peft}{PEFT}{Parameter-efficient Fine-tuning}
\newacronym{lora}{LoRa}{Low-Rank Adaptation}
\newacronym{cot}{CoT}{chain-of-thought}
\newacronym{obq}{OBQ}{Optimal Brain Quantization}
\newacronym{cqu}{CQU}{Contextual Query Understanding}
\newacronym{conrag}{ConRAG}{Conversational Retrieval-Augmented Generation}
\newacronym{fid}{FiD}{Fusion-in-Decoder}
\newacronym{hr}{HR}{Hit-Ratio}
\newacronym{mrr}{MRR}{Mean Reciprocal Rank}
\newacronym{bleu}{BLEU}{Bilingual Evaluation Understudy}
\newacronym{rogue}{ROGUE}{Recall-Oriented Understudy for Gisting Evaluation}
\newacronym{mahueval}{MaHuEval}{manual human evaluation}
\newacronym{auev}{AutoEval}{automatic evaluation}
\newacronym{er}{ER}{Examination Regulation}
\newacronym{poc}{PoC}{proof-of-concept}
\newacronym{leolm}{LeoLM}{Linguistisch Erweitertes Offenes Language Model}
\newacronym{cast}{CAsT}{Conversational Assistance Track}


\usepackage{amsmath} % für \@addtoreset
\makeatletter
\@addtoreset{definition}{chapter}
\makeatother
\newtheorem{definition}{Definition}[chapter]

\glsaddall

\begin{document}

% TITLE:
\pagenumbering{roman} 
\begin{titlepage}


\vspace*{1cm}
\begin{center}
\vspace*{3cm}
\textbf{ 
\Large Heidelberg University\\
\smallskip
\Large Institut of Computer Science\\
\smallskip
\Large Data Science Group\\
\smallskip
}

\vspace{3cm}

\textbf{\large Master's Thesis} % Bachelor-Arbeit 

\vspace{0.5\baselineskip}
{\huge
\textbf{Building Conversational Question} \vspace{0.2cm}\\
\textbf{Answering Systems}
}
\end{center}

\vfill 

{\large
\begin{tabular}[l]{ll}
Name: & Stephan Lenert\\
Matriculation Number: & 3734583\\
Supervisor: & Prof. Dr. Michael Gertz\\
Submission Date: & \today
\end{tabular}
}

\end{titlepage}

\onehalfspacing

\thispagestyle{empty}

\vspace*{100pt}
\noindent
Hiermit versichere ich, dass ich die Arbeit selbst verfasst und keine anderen als die angegebenen Quellen und Hilfsmittel benutzt und wörtlich oder inhaltlich aus fremden Werken Übernommenes als fremd kenntlich gemacht habe. Ferner versichere ich, dass die übermittelte elektronische Version in Inhalt und Wortlaut mit der gedruckten Version meiner Arbeit vollständig übereinstimmt. Ich bin einverstanden, dass diese elektronische Fassung universitätsintern anhand einer Plagiatssoftware auf Plagiate überprüft wird.
\vspace*{50pt}
\noindent

\underline{\phantom{mmmmmmmmmmmmmmmmmmmm}}

\medskip
\noindent 
Abgabedatum: \today
\newpage

% Add a brief summary of your topic and contributions (Zusammenfassung) in German *and* in English:
\chapter*{Zusammenfassung}
Das Interesse an der Entwicklung von Systemen für die konversationelle Informations\-suche hat durch den Boom von ChatGPT und Large Language Modellen im generellen stark zugenommen. Jedoch ist eine alleinige Verwendung von ChatGPT aufgrund von Herausforderungen wie der Aktualisierung und Veränderung von Wissen, Halluzinationen oder der Interpretierbarkeit von KI nicht ausreichend. Die Bewältigung dieser Hürden erfordert die Entwicklung von Retriever-Reader-Systemen. Allerdings ist die Konstruktion solcher Systeme für reale Anwendungsfälle in der aktuellen Forschung noch nicht ausreichend abgedeckt. Diese Arbeit schlägt ein umfassendes Rahmenwerk für Retrieval-Augmented Generation (RAG)-basierte konversationelle Frage-Antwort-Systeme vor, das vier Komponenten umfasst: Extractor, Contextual Query Understanding (CQU), Retriever und Reader. Es werden das Problemfeld, die Herausforderungen und Lösungsansätze für jede Komponente beschrieben, wobei der Fokus darauf liegt, bestehende Retriever- oder Reader-Modelle zu nutzen, anstatt neue Architekturen zu entwickeln. Für die Herausforderungen im Bereich des Retrievers werden Lösungen vorgeschlagen, die auf synthetischen Daten für Probleme im Zusammenhang mit dem Datensatz im generellen und verschiedenen Methoden wie Mixture-of-Experts für Probleme im Zusammenhang mit dem Evidence Set basieren. Für die Extract-Komponente wird eine pipelinefähige Kombination von Operationen vorgeschlagen. Die Aufgabe des Readers wird in mehrere kleinere Herausforderungen unterteilt, für die Lösungsansätze basierend auf Fine-Tuning oder Nachbearbeitung des Evidence Sets entwickelt werden. Unter Verwendung dieses Frameworks wird ein beispielhafter Frage-Antwort-Chatbot für die Prüfungsordnung der Universität Heidelberg entwickelt. Die Evaluation dieses Chatbots zeigt Engpässe in den Komponenten Retriever und Extractor auf, hebt die Nachteile kleinerer Sprachmodelle wie Llama2-7B-chat im Vergleich zu größeren Modellen wie gpt-3.5-turbo hervor und unterstreicht die Grenzen synthetischer Daten für die automatische Evaluation, die vor allem nur für faktoide Fragen anwendbar sind.
\newpage

\chapter*{Abstract}
The resurgence of interest in conversational information seeking, reignited by the boom of systems like ChatGPT and large language models in general, faces challenges such as knowledge extension, hallucination, and AI interpretability. Overcoming these hurdles necessitates the development of retriever-reader systems, yet constructing such systems from scratch for real-world use cases remains underexplored in current research. This thesis proposes a comprehensive framework for retrieval-augmented generation (RAG)-based conversational question-answering, comprising four components: Extractor, Contextual Query Understanding (CQU), Retriever, and Reader. We delineate the problem field, challenges, and solutions for each component, focusing on leveraging existing Retriever or Reader models rather than reinventing them. We address challenges in the Retriever, proposing solutions based on synthetic data for knowledge source challenges and various methods such as Mixture-of-Experts for evidence set challenges. For the Extract component, we propose a pipeline combination of operations. The Reader's task is broken down into multiple micro challenges and proposes solutions for them based on fine-tuning or post-processing of the evidence set. Applying this framework, we develop an exemplary question-answering chatbot for Heidelberg University's examination regulations. Evaluation uncovers bottlenecks in the Retriever and Extractor components, highlights the drawbacks of smaller language models like Llama2-7B-chat compared to larger models like gpt-3.5-turbo, and emphasizes the limitations of synthetic data for automatic evaluation, primarily applicable to factoid questions. 
\newpage

% MAIN PART:
% Table of contents (Inhaltsverzeichnis)
\tableofcontents
\cleardoublepage

% List of acronyms (Abkürzungsverzeichnis):
% \chapter*{List of Acronyms}
\printglossary[type=\acronymtype, title=List of Acronyms]

% List of figures (Abbildungsverzeichnis):
\listoffigures
% List of tables (Tabellenverzeichnis):
\listoftables

%%%%%%%%%%%%%%%%%%%%%%%%%%%%%%%%%%%%%%%%%%%%%%%%%%%%%%%%%%%%%%%
% Here, the actual content of your thesis begins
% You can either put all the text here or use individual files to store the chapters of your thesis.
% Below are templates for both alternatives.


\chapter{Introduction}
\label{chap:intro}
\pagenumbering{arabic} 
\section{Motivation}

The journey of exploring natural language querying dates back to as early as 1961, with researchers embarking on projects like Baseball \cite{green_baseball_1961}, a program designed to respond to users' natural language queries within the domain of baseball. In 1999, the \gls{trec} initiated the \gls{trec}-8 Question Answering track, marking \enquote{the first large-scale evaluation of domain-independent question-answering systems} \cite{voorhees_trec-8_1999}. A more renowned \gls{qa} system is IBM's \textit{Watson}, an open-domain \gls{qa} system that famously triumphed on the television game show Jeopardy! in 2011 \cite{ferrucci_introduction_2012}. The advent of conversational assistants such as Alexa, Siri, and Cortana further fueled interest in conversational information seeking. In 2019, the establishment of the \gls{trec} \gls{cast} aimed to nurture \gls{cis} as an active research field and provide large-scale reusable test beds. However, the true surge in user engagement came with the release of ChatGPT by OpenAI in November 2022, which achieved an astounding one million users within five days \cite{demandsage2022chatgpt}.

Since the widespread adoption of ChatGPT, interest in conversational information seeking has surged once again. However, using a generative \gls{llm} only has certain drawbacks, including the following:

\begin{enumerate}
    \item Parameterized Knowledge: Expansion or updating of knowledge is challenging.
    \item Explainability: Understanding the rationale behind predictions is difficult.
    \item Hallucination: The model may invent seemingly factual information not present in the underlying knowledge source.
\end{enumerate}

These limitations render the sole use of generative models inadequate for practical conversational information-seeking use cases. Consequently, researchers have explored various approaches to address these issues and develop conversational information-seeking systems that mimic human-like conversations \cite{ferrucci_introduction_2012,guu_realm_2020,lewis_retrieval-augmented_2021,nakano_webgpt_2022}. Among these approaches, \gls{rag} \cite{lewis_retrieval-augmented_2021} has emerged as a leading solution, evident in trending frameworks like Langchain \cite{noauthor_question_nodate} and even ChatGPT itself \cite{noauthor_chatgpt_2023}.

However, despite advancements in research, there is a notable absence of papers detailing the adoption of these techniques in real-world use cases. To our knowledge, only two such papers exist \cite{feng_dialdoc_2021,gholami_zero-shot_2021}, which do not comprehensively address the entire problem domain associated with this task, such as the extraction step, and fail to leverage modern possibilities with \gls{llm}s.

\section{Objectives and Contributions}

This thesis aims to bridge this gap by addressing the question of how to build a faithful conversational question-answering system for real-world use cases, a topic that has not been adequately explored in scientific literature to date. This thesis resolves this issue by firstly laying out the fundamental concepts and techniques necessary, secondly developing a holistic framework for constructing such a system based on \gls{rag}, and thirdly implementing a \gls{poc} based on the examination regulations of Heidelberg University. Finally, we evaluate the \gls{poc} using synthetic data and human evaluation. The core contributions of this thesis are as follows:

\begin{itemize}
    \item Providing a holistic overview of the challenges in conversational question-answering systems tailored to a specific document collection and developing a comprehensive framework for constructing such systems based on the four components: Extract, CQU, Retriever and Reader.
    \item Breaking down the extraction of passages from a document collection into pipeline operations.
    \item Identifying challenges faced by the Retriever component towards both the knowledge base and the evidence set and how to resolve them using synthetic-data-based fine-tuning and Mixture-of-Experts.
    \item Decomposing the task of the Reader component into micro-challenges and proposing solutions for them.
    \item Offering multiple approaches for evaluating conversational question-answering systems, with the major insight, that the component-wise evaluation for non-factoid questions needs new approaches and currently at least in the case of this thesis, the human end-to-end evaluation was the most robust approach.
    \item Highlighting the limitations of synthetic data for system evaluation, as those can only be used for factoid question generation due to the used pattern of question-context-answer tuples.
    \item Providing insights into the current bottlenecks of use-case-specific \gls{convqa} systems. Mainly the retriever component and the associated extraction of passages are drawbacks. In the end-to-end evaluation, 46.5\% of errors were attributed to the retriever component and 18.5\% to the extraction component directly.
\end{itemize}

Those cover the main contributions and insights of this thesis, for detailed insights, please refer to the respective chapters. 

\section{Thesis Structure}

The construction of a conversational question-answering system is approached in two stages within this thesis. Firstly, we focus on question-answering based on a single natural language query, followed by the addition of the conversational component. Thus, Chapter \ref{chap:grundlagen} provides a comprehensive overview. Section \ref{sec:qa} presents various approaches and concepts in the field of question-answering, while Section \ref{sec:cqa} extends this perspective to include conversational elements. Narrowing down the range of potential solutions, Chapter \ref{chap:main} introduces the \gls{conrag} framework, which lays out the spectrum of challenges, from handling document collections to enabling conversational question-answering. It's important to note that we streamline the choice of possible implementations to a system consisting of four distinct components: Extract, CQU, Retriever and Reader. The validation of this newly established system approach is conducted in a \gls{poc} manner in Chapter \ref{chap:eval}. Here, we implement and evaluate an exemplary system using the collection of examination regulations from Heidelberg University. This chapter also includes insightful reflections on the evaluation of conversational question-answering systems. This concludes the content outline of this thesis.


%%%%%%%%%%%%%%%%%%%%%%%%%%%%%%%%%%%%%%%%%%%%%%%%%%%%%%%%%%%%
\newpage
%%%%%%%%%%%%%%%%%%%%%%%%%%%%%%%%%%%%%%%%%%%%%%%%%%%%%%%%%%%%

\chapter{Background and Related Work}
\label{chap:grundlagen}

This chapter provides essential background information and reviews relevant prior research. It commences with an introduction to the sub-task of \gls{qa}, as presented in Section \ref*{sec:qa}. As mentioned in Chapter \ref{chap:intro}, this chapter maintains a clear distinction between \gls{qa} and \gls{convqa}. Consequently, Section \ref{sec:cqa} extends upon the foundational knowledge of \gls*{qa} and introduces the requisite concepts for the transformation of a QA-System into a \gls{convqa}-System. Section \ref{sec:efficient_llm} elaborates on different approaches towards the application of \gls{llm}s in resource-constrained settings. Section \ref{sec:related_work} delves into the related work, providing a comprehensive overview of the current state-of-the-art in the field of \gls{qa} and \gls{convqa} over textual knowledge sources. The goal of this chapter is to provide a holistic foundation of the research field of \gls{qa} and \gls{convqa}. Not all mentioned concepts and methods will further be utilized in Chapters \ref{chap:main} and \ref{chap:eval} but are a necessary part of the research to understand the advantages and limitations of modern approaches and system design decisions. The introductions to certain sections will elaborate on the relevance of the respective topics for the research of this thesis.

\section{Question Answering}
\label{sec:qa}

In Section \ref{subsec:qa_basics} we will lay the groundwork by introducing the fundamental aspects of \gls{qa}-Systems and the techniques used to differentiate and categorize them. Following that, subsequent sections will delve deeper into the examination of specific system components.

The concepts and taxonomy of \gls{qa} systems mentioned in Section \ref{subsec:qa_basics} are essential background knowledge. While Section \ref{subsec:qa_architectures} outlines different conceptual approaches towards the holistic architecture of \gls{qa}-systems, this thesis will later focus solely on \gls{rag}s. The following sections, \ref{subsec:qa_indexing}, \ref{subsec:qa_retrieval}, and \ref{subsec:qa_reader}, will give a holistic image of possible approaches towards these components, which is necessary to understand on a higher level what possible system designs entail. However, the thesis will mainly focus on zero-shot retrievers, their optimization, and generative readers. The other mentioned methods can be viewed as possible alternatives and historical developments that should not be forgotten in the research process. The decision to include this information was motivated by the dynamic nature of the research field of \gls{qa}. While certain methods may have gained popularity in recent months, it's essential to recognize that the landscape of QA is continually evolving. Just because a particular method has gained traction does not render other approaches obsolete. By acknowledging the ongoing development and diversity within the field, we aim to provide a comprehensive understanding and context in this thesis.


\subsection{Basics}
\label{subsec:qa_basics}

Jurafsky and Martin define a \gls{qa}-System as a system \enquote{designed to satisfy human information needs} \cite{jurafsky_speech_2023}. Hence, it primarily functions as an Information Retrieval System, with its primary objective being to provide users with the desired and accurate information in response to natural language requests.

The research community has yet to establish a universally accepted classification framework for \gls{qa} systems. For instance, Hao et al. and Farea et al. \cite{hao_recent_2022, farea_evaluation_2022} take a comprehensive approach to classify QA systems but differ in certain aspects, such as their treatment of question types and knowledge sources. On the other hand, other researchers \cite{zhu_retrieving_2021, jurafsky_speech_2023, etezadi_state_2023, zhang_survey_2023} employ a similar classification methodology but often focus solely on retrieval-based approaches, thereby lacking a holistic perspective.

The classification proposed by Farea et al. \cite{farea_evaluation_2022} goes a step further by distinguishing between the \textbf{QA-Framework} and \textbf{QA-Paradigms}, enhancing its versatility for comparing classical and modern QA systems. An adaptation of this classification will be utilized in this thesis. The originally proposed QA algorithms have been extended to include the Retrieval-based approach, and the Question Types have been revised based on the typology introduced by Mishra et al. in their 2016 survey \cite{mishra_survey_2016}, which was further elaborated upon by Etezadi et al. \cite{etezadi_state_2023}. Also the answer types were adjusted to align with the classifications used in \cite{mcdonald_detect_2022,dasigi_dataset_2021}. In this context, a crucial distinction is made between a \textbf{QA} and \textbf{ConvQA} system, guided by the criteria outlined in \cite{zamani_conversational_2023}: a \textbf{QA} system exclusively handles standalone questions, while any inquiry exceeding a single question and involving conversational context falls within the domain of a \textbf{ConvQA} system.

The \textbf{\gls{qa}-Framework} encompasses external factors such as Question and Answer Types, while also considering system-related factors like the \gls{qa} Algorithm and Knowledge Source \cite{farea_evaluation_2022, hao_recent_2022}. Conversely, the \textbf{\gls{qa}-Paradigm} defines the fundamental underlying concept of a system and can be seen as a subset of possible combinations within the \textbf{\gls{qa} Framework}. Currently, three dominant paradigms prevail:

\begin{enumerate}
    \item \textbf{Information Retrieval (IR)-Based QA}: This paradigm involves searching through extensive multi-modal data based on a user's question and using the retrieved passages to generate an answer.
    
    \item \textbf{Knowledge Base (KB) QA}: In this approach, a semantic representation of the question is constructed, and a knowledge base is queried using this representation. The returned results are then used to generate an answer.
    
    \item \textbf{Generative Question Answering}: Here, knowledge is fully implicit, and a \gls{nn} generates answers based on its trained parameters.

\end{enumerate}

For visual clarity, a diagram illustrating the adjusted \gls{qa} Framework Classification by Farea et al. is provided in Figure \ref{fig:qa_classification}.

\begin{figure}[h]
    \centering
    \includegraphics[width=\textwidth]{Grafiken/QA_Framework.png}
    \caption{Adjusted QA Framework Classification by Farea et al. \cite{farea_evaluation_2022}}
    \label{fig:qa_classification}
\end{figure}


Figure \ref{fig:qa_classification} illustrates the aforementioned classification. The primary distinguishing factor is the employed \textbf{\gls{qa} Algorithm}. Rule-based approaches involve the manual crafting of feature extractions from user questions, which are then compared to the knowledge base. Rule-based approaches are typically employed in closed-domain \gls{qa} systems exclusively \cite{etezadi_state_2023}.

Retrieval-based approaches are the classic Information Retrieval (IR)-based \gls{qa} systems, comprising two key components: an intent classifier and a retriever. The intent classifier's objective is to discern the question's intent and identify important entities. Subsequently, the retriever searches the knowledge source and identifies the most relevant passages \cite{farea_evaluation_2022, zhu_retrieving_2021}.

The Neural-based approach, often referred to as the generative approach, utilizes a Sequence-to-Sequence (S2S) model to generate accurate answers to given questions. In this paradigm, the information is stored directly in the neural network's parameters, otherwise, the neural network is part of a Retrieval-based approach. Most datasets in these contexts consist of triples of question, context, and answer pairs \cite{jurafsky_speech_2023}. Notably, widely used datasets such as SQuAD and QASPER originally emerged from the field of machine reading comprehension, representing a foundational step in the evolution of \gls{qa} systems \cite{rajpurkar_squad_2016, dasigi_dataset_2021, zhu_retrieving_2021}.

In addition to the \textbf{\gls{qa} Algorithms}, the \textbf{Knowledge Source} plays a pivotal role in distinguishing various aspects of Question Answering (QA) systems. The nature of the knowledge source can range from structured to unstructured or semi-structured, and it may encompass diverse data modalities, including text, audio, and video. A common point of comparison in the QA landscape is between closed and open-domain systems.

In the broad sense, a \textbf{closed-domain} QA system operates within the confines of a specific knowledge domain, which means it has limited access to information. In contrast, \textbf{open-domain} QA systems grapple with an extensive array of knowledge sources, necessitating a more versatile approach \cite{farea_evaluation_2022}.

Furthermore, a closed-domain setup often entails limitations on the types of questions it can handle, primarily focusing on factoid questions or predefined templates. Additionally, it frequently relies on structured knowledge bases like graphs or logically organized repositories \cite{hao_recent_2022}.

Conversely, open-domain QA systems are designed to tackle a wide spectrum of user queries, ranging from factoids to more complex inquiries. They typically deal with unstructured knowledge sources, which can be substantial and diverse in content \cite{zhu_retrieving_2021, farea_evaluation_2022, jurafsky_speech_2023}.

An alternative perspective for distinguishing \gls{qa}-Systems lies in the \textbf{Question Types} that users can input into the system. Questions can fall into various categories, such as \textit{factoid, list, casual, confirmation, hypothetical} \cite{mishra_survey_2016}, or \textit{complex}~\cite{etezadi_state_2023}.\nopagebreak

\begin{itemize}
   \item \textit{Factoid questions}, the most common type, is typically signaled by question words (what, when, which, who, how) and yields a concise factual answer.
   
   \item \textit{List questions} represent a specialized subset of factoid questions, where the answer comprises a list of facts.
   
   \item \textit{Casual questions} encompass inquiries that deviate from the factoid format, often involving words like \textit{how} or \textit{why} and requiring more advanced reasoning.
   
   \item \textit{Confirmation questions} seek simple yes or no responses, frequently employed in personal assistant applications.
   
   \item \textit{Hypothetical questions} delve into hypothetical scenarios (e.g., "what would happen if"), aiming for plausible rather than definitive answers.
   
   \item \textit{Complex questions} can be further categorized into \textit{answer-retrieval-complex} and \textit{question-understanding-complex}. In the case of question-understanding-complex questions, the complexity arises from nuances like multiple constraints, making the question itself intricate to comprehend. In contrast, answer-retrieval-complex questions involve complexities in finding the correct answer, often requiring the combination of information from multiple documents or similar sources. This is commonly referred to as long-form \gls{qa}.
\end{itemize}

Lastly, a \gls{qa}-System can be characterized by the \textbf{Answer Types} it offers, a concept closely intertwined with Question Types. Farea et al. \cite{farea_evaluation_2022} delineate four categories of answers: \textit{extractive, abstractive, boolean} and \textit{reactive}. 

\begin{itemize}
   \item \textit{Extractive answers} represent the most common type, where the answer is a specific factual excerpt presented as a span of tokens.

   \item \textit{Abstractive answers} typically correspond to complex questions that necessitate the system to consider multiple documents and information sources to formulate a response. In such cases, no predefined or annotated answer exists.

   \item \textit{Boolean answers} are typically the result of confirmation questions, where the answer is either \textit{yes} or \textit{no}.

   \item \textit{Reactive answers} often arise in response to confirmation questions and can be a system-generated reaction based on the user's provided answer.
\end{itemize}

% Pushed 1st

\subsection{Information Retrieval Architectures}
\label{subsec:qa_architectures}

As stated in the previous section, there are three major paradigms in \gls{qa}: \gls{ir}-based \gls{qa}, \gls{kb}-based QA, and Generative \gls{qa}. This section will primarily concentrate on the first paradigm, \gls{ir}-based QA, as it holds the most promise for addressing the objectives of this thesis topic.

This thesis will not focus on \gls{kb} QA, as this approach requires the mapping of the query to a structured data representation. As the task of this thesis is to develop a general system, that is adaptable to different data inputs, \gls{kb} QA will be excluded \cite{dimitrakis_survey_2020} (See Section \ref{sec:overview}).

Generative \gls{qa} is often denoted as \textit{Retriever-free} or \textit{Neural-based} approaches. The central characteristic of this paradigm is that knowledge resides within the parameters of a neural network. Consequently, the knowledge is implicit, and the \gls{qa} system will not furnish a specific document, passage, or other source from which it extracted the information. Instead, it offers a textual excerpt. While these systems can achieve competitive performance compared to \gls{ir}-based \gls{qa} systems, they are not under consideration for this thesis due to their lack of reference, which is a crucial requirement for the system (See Section \ref{sec:overview}) to be developed \cite{roberts_how_2020}.


\begin{figure}
    \centering
    \includegraphics[width=\textwidth]{Grafiken/Retriever_Reader.png}
    \caption{Reader-Retriever-System Architecture for QA by Zhu et al. \cite{zhu_retrieving_2021}.}
    \label{fig:rr_architecture}
\end{figure}

Figure \ref{fig:rr_architecture} depicts the general architecture of a \textbf{Retriever-Reader-System}, as defined by Zhu et al. \cite{zhu_retrieving_2021}. This architecture serves as the foundational framework for \gls{ir}-Based \gls{qa} systems and was initially introduced by Harabagiu et al. \cite{harabagiu_open-domain_2003}. In this framework, all modules operate independently, can be trained separately, and are subject to independent evaluation.

The \textbf{Retriever} module's primary role is to retrieve relevant documents, passages, or other pertinent information from a knowledge source and rank them based on their relevance to answering the user's query. Subsequently, the \textbf{Reader} module extracts the answer from the retrieved documents and presents it to the user. This task bears a close resemblance to \gls{mrc}, with the key distinction that in \gls{ir}-Based \gls{qa}, the system must handle multiple documents and comprehend them to formulate a response, unlike classical \gls{mrc} tasks, which typically involve only one context document.

The \textbf{Document Post-Processor} module's role is to curate and refine the set of documents that will be forwarded as \textit{Relevant Documents} to the subsequent stage, the Reader. Concurrently, the \textbf{Answer Post-Processor} assists the Reader in addressing complex questions for which the answer may not be found in a single document alone \cite{zhu_retrieving_2021,jurafsky_speech_2023}.

It's worth noting that some researchers include a \textbf{Question Analysis} module preceding the Retriever, which aims to preprocess the received question for more efficient query execution in the Retriever \cite{nassiri_transformer_2023}. However, for the purposes of this thesis, we adhere to Zhu et al.'s definition \cite{zhu_retrieving_2021}, where this functionality is considered part of the Retriever.

Conceptually, there are three distinct approaches to the Retriever itself: \textit{Sparse Retrieval, Dense Retrieval,} and \textit{Iterative Retrieval.} The specifics of these approaches will be thoroughly explored in Section \ref{subsec:qa_retrieval}.

Document Post-Processors can be categorized into \textit{Supervised Learning, Reinforcement Learning,} and \textit{Transfer Learning}-based approaches. A detailed discussion of these approaches is also provided in Section \ref{subsec:qa_retrieval}.

In Section \ref{subsec:qa_reader}, we will delve into the finer details of Reader approaches and Answer Post-processing. Broadly speaking, there are two primary types of Readers: \textit{Extractive} and \textit{Generative} Readers. As for Answer Post-processing, it involves two key categories: \textit{Rule-based} and \textit{Learning-based} approaches.

There are also \textbf{End-to-End} approaches that employ a single module to execute the entire \gls{qa} task. Excluding generative approaches, two common categories of such approaches are \textbf{Retriever-Reader} and \textbf{Retriever-only} models.

An End-to-End Retriever-Reader aims to train both the Retriever and Reader in a single backpropagation step, and in some cases, it introduces additional knowledge sources beyond the traditional \gls{ir} framework. An illustrative example is \gls{rag} \cite{lewis_retrieval-augmented_2021}. \gls{rag} consists of a pre-trained Generator with implicit knowledge encoded in its parameters and a pre-trained Retriever. For each question, the Retriever identifies the most relevant documents and generates a latent vector based on them. This latent vector, along with the original question, is fed into the Generator. Section \ref{subsec:qa_reader} will delve into details regarding the \gls{rag} architecture.

Another end-to-end approach, similar to \gls{rag}, is \gls{realm} \cite{guu_realm_2020}. While these previous two approaches extended the capabilities of pre-trained \gls{s2s} models, Nishida et al. pursued a different path by training a single \gls{nn} to perform both tasks simultaneously: \gls{ir} and \gls{mrc} \cite{nishida_retrieve-and-read_2018}.

It is noteworthy that all these end-to-end approaches have demonstrated competitive performance compared to state-of-the-art methods on specific \gls{qa} datasets.

An essential yet often underestimated question is: What defines textual data, and how should one preprocess formats such as PDFs to extract this textual content? While many datasets already comprise small contextual snippets \cite{wang_modern_2022}, it's crucial not to overlook the entire process of extracting snippets from unstructured PDFs, for example. Approaches to tackle this challenge will be explored in detail in the upcoming Section \ref{subsec:qa_indexing}.

% 2nd push

\subsection{Extraction Approaches}
\label{subsec:qa_indexing}

As discussed in the previous Section \ref{subsec:qa_basics}, the knowledge source for a \gls{qa}-System can take the form of textual or multimodal data. The specific type of data may necessitate certain requirements or specific adjustments to the Retriever used for \gls{ir}.

In the context of this thesis, the primary knowledge source to be employed is PDF documents (See Section \ref{sec:overview}). In the research field, three major approaches exist for extracting textual information from unstructured data types like PDFs: \textit{visual} \cite{tito_document_2021}, \textit{direct} \cite{wang_multi-passage_2019}, and \textit{alternative} \cite{dasigi_dataset_2021} extraction methods.

It's important to note upfront that the chosen extraction method is intricately connected to the subsequent retrieval approach. The specifics, including metadata alongside pure textual data and quality requirements, may vary among different extraction and retrieval methods.

The visual approach is closely aligned with the research field of \textit{Document Question Answering}. A well-known example dataset in this field is \gls{docvqa} \cite{tito_document_2021}. The primary concept behind the visual approach to document question-answering is to capture not only the text of a PDF but also additional information such as the document's structure, various hierarchies on a page (e.g., sections, subsections), and the ability to analyze tables and figures. These hierarchical structures can be leveraged to create two-stage retrieval approaches. In these approaches, initially, a collection of relevant files is identified based on higher-level attributes like the document's title and abstract. Subsequently, a more granular retrieval process is executed over lower-level attributes such as passages within the relevant files. This \textit{Iterative Retrievers} will be further discussed in Section \ref{subsec:qa_retrieval} \cite{liu_dense_2021}.

The challenge of \textit{Visual Document Question Answering} typically involves taking images of PDF pages as inputs and mapping question-answer pairs to them. The answers are extracted from either a single paragraph or a combination of multiple paragraphs \cite{mathew_document_2021}. Nonetheless, the extraction pipeline in this case usually resembles the \textit{Retriever-Reader} architecture, where the extracted information from the visual processing is fed into such a system afterward. Researchers in this field often employ a pipeline that includes a \textit{Document Layout Analysis} model, followed by the application of an \gls{ocr} tool to the detected regions \cite{mcdonald_detect_2022}. Examples of a \textit{Document Layout Analysis} model include the Document Image Transformer by Li et al. \cite{li_dit_2022}.

The direct approach is the most prevalent method in the field of Question Answering (\gls{qa}) and Information Retrieval (\gls{ir}). The primary concept behind this approach is to extract textual information from PDFs and store it in a database. The extraction process can be accomplished using various tools such as \textit{PDFMiner} or \textit{Adobe Extract} \cite{meuschke_benchmark_2023}. However, a lingering question is how to effectively split the extracted textual data, especially considering that they are often not cleaned after extraction.

A common practice when employing a Language Model (\gls{llm}) is to optionally cleanse the text corpus and then divide it based on a predefined token size. This approach is evident in two notable open-source LLM projects: \textit{Langchain} and the \textit{Retrieval Plugin for ChatGPT} by OpenAI \cite{noauthor_langchain-ailangchain_nodate,noauthor_chatgpt_2023}. In the original Dense Retrieval paper by Karpukhin et al., a sliding window of token size 5 was utilized \cite{karpukhin_dense_2020}. Therefore, it can be assumed that for contemporary LLM applications, the precise quality of the data, ensuring that a document contains syntactically correct sentences, may not be as critical.

Apart from modern approaches involving text clipping, previous methods aimed to identify paragraphs and similar structures within the extracted texts \cite{zhu_retrieving_2021}.

An alternative approach involves the methodology employed in constructing the QASPER dataset. In this case, the authors conducted a pre-filtering of scientific papers' PDFs, selecting only those with freely accessible LaTeX files. They then utilized the S2ORC tool to extract cleaned textual data from these LaTeX files \cite{dasigi_dataset_2021}. It's important to note that this approach is highly specific to the QASPER dataset and cannot be universally applied. Nonetheless, it serves as an illustration of alternative methods for extracting textual data from PDFs.

% Commit #3

\subsection{Retrieval Approaches}
\label{subsec:qa_retrieval}

The traditional state-of-the-art in \gls{ir} relies on \textbf{Sparse Retrievers}, with one notable example being BM25. BM25 is renowned as "one of the most empirically successful retrieval models and is widely used in current search engines" \cite{zhu_retrieving_2021}. Nandan et al. even demonstrated that on modern \gls{odqa} datasets, BM25 remains a viable baseline for zero-shot \gls{ir} \cite{thakur_beir_2021}.

BM25 was originally introduced by Robertson et al. \cite{robertson_probabilistic_2009}. It operates by utilizing the TF-IDF token weights between a question $q$ containing tokens $q_1, \ldots, q_T$ and a set of passages $P$, where $p \in P$.

\begin{equation}
    \mathbf{s}_{q, p}^{\text{BM25}}=\sum_{i=1}^T \log \left(\frac{|\mathcal{P}|}{N\left(q_i, \mathcal{P}\right)}\right) \frac{n\left(q_i, p\right)\left(k_1+1\right)}{k_1\left(1-b+\frac{b|p|}{a v p l}\right)+n\left(q_i, p\right)}
    \label{eq:bm25}
\end{equation}

    
Equation \ref{eq:bm25} illustrates the BM25 score for a question $q$ and a passage $p$. In this equation, $N\left(q_i, \mathcal{P}\right)$ represents the count of passages in $\mathcal{P}$ that contain the token $q_i$, while $n\left(q_i, p\right)$ indicates the frequency of token $q_i$ within the passage $p$. The variable $|p|$ signifies the length of passage $p$, and $avpl$ stands for the average passage length in $\mathcal{P}$. The parameters $k_1$ and $b$ are free parameters, typically set to $k_1 = 0.9$ and $b = 0.4$ \cite{mcdonald_detect_2022,robertson_probabilistic_2009}.

Traditionally, this lexical Information Retrieval (\gls{ir}) approach has been capable of providing satisfactory retrieval results. However, in 2020, Karpukhin et al. demonstrated for the first time that a \textbf{Dense Retrieval} approach could outperform the Sparse Retrieval approach across multiple \gls{odqa} datasets \cite{karpukhin_dense_2020}. Consequently, the search for a general Dense Retrieval model has been ongoing, as these Dense Retrieval approaches offer advantages such as semantic matching and the ability to handle lengthy documents.

In general, there are three types of Dense Retrieval approaches \cite{zhu_retrieving_2021}: the \textbf{Representation-based Retriever}, often referred to as the \textit{dual-encoder} \cite{karpukhin_dense_2020}; the \textbf{Interaction-based Retriever}, often referred to as the \textit{cross-encoder}; and the \textbf{Representation-interaction Retriever}, often referred to as the \textit{multi-stop retriever}. Figure \ref{fig:types_of_retriever} illustrates the general architecture of these three types of Dense Retrievers.

The \textbf{Dense Passage Retriever (DPR)} by Karpukhin et al. serves as a notable example to explain the \textbf{Representation-Based Retriever}. Given a collection $M$ of text passages $p$ and a question $q$, the objective of DPR is to identify the $k$ most similar passages to the question. To achieve this, DPR employs two distinct \textbf{BERT} \cite{devlin_bert_2019} Encoders. One Encoder, denoted as $E_Q(\cdot)$, encodes the question $q$ into a $d$-dimensional vector, where $d = 768$. The other Encoder, labeled as $E_P(\cdot)$, encodes the passage $p$ into a $d$-dimensional vector at the \verb|[CLS]| token. The similarity between these two vectors is computed using the inner product:


\begin{equation}
    \mathbf{s}_{q, p}^{D P R}=\mathbf{E}_{Q}(q)^{\top} \mathbf{E}_{P}(p)
    \label{eq:dpr}
\end{equation}

The choice of the inner product as the similarity function is motivated by its computational efficiency and the demonstrated, comparable performance \cite{karpukhin_dense_2020}. The dot-product must yield a small value for pairs of questions and passages that are genuinely related. The training dataset $D$ comprises $m$ instances, where $q_i$ represents the question, $p_i^+$ denotes the positive passage, and $p_{i,n^-}$ represents the negative passage:

\begin{equation}
    \mathbf{D}=\left\{\left(q_{i}, p_{i}^{+}, p_{i, 1}^{-}, \ldots, p_{i, n}^{-}\right)\right\}_{i=1}^{m}
\end{equation}

The loss function is optimized using the negative log-likelihood of $p_i^+$:

\begin{equation}
    \mathcal{L}_{D P R}=-\log \frac{\exp \left(\mathbf{s}_{q_i,p_i^{+}}^{D P R}\right)}{\exp \left(\mathbf{s}_{q_i,p_i^{+}}^{D P R}\right) + \sum_{j=1}^{n} \exp \left(\mathbf{s}_{q_i,p_{i,j}^{-}}^{D P R}\right)}
\end{equation}

It's important to note that in \cite{karpukhin_dense_2020}, the selection of negative passages was not arbitrary. Instead, two additional approaches were employed: BM25 top passages that do not contain the answer and positive passages paired with other questions.

One significant advantage of the Representation-Based Retriever is that passages can be pre-indexed locally rather than at runtime. This reduction in latency between the question and the response may, however, come with trade-offs in the quality of the retrieved passages.

The \textbf{Interaction-Based Retriever} incorporates both the question $q$ and the passage $p$ within a single model, separated by a \verb|[SEP]| indicator. These models offer various approaches for modeling the relationship between $q$ and $p$. For instance, one common method is to utilize the \verb|[CLS]| classifier as an indicator of whether the passage is relevant to the question. This approach was first introduced with \gls{bert} \cite{devlin_bert_2019}. While these models perform competitively with previous Representation-Based Retrievers, it's important to note that they are 100-1000 times more computationally expensive \cite{khattab_colbert_2020}.

To address this latency issue, models like ColBERT introduced the concept of \textbf{co}ntextualized \textbf{l}ate interaction \cite{khattab_colbert_2020}. In this thesis and subsequently in research, it is referred to as the \textbf{Representation-Interaction Retriever} \cite{zhu_retrieving_2021}.


\begin{figure}
    \centering
    \includegraphics[width=\textwidth]{Grafiken/Types_of_Retriever.png}
    \caption{Types of Dense Retriever by Zhu et al. \cite{zhu_retrieving_2021}.}
    \label{fig:types_of_retriever}
\end{figure}

ColBERT, like \gls{dpr}, employs two \gls{bert} Encoders, denoted as $E_Q\left(\cdot\right)$ and $E_P\left(\cdot\right)$. However, it introduces a late interaction mechanism. When provided with a query $q$, it is initially tokenized into BERT-based Wordpiece tokens, resulting in $q_1, \ldots, q_T$. Following the \verb|[CLS]| token, a \verb|[Q]| token is appended to signify the question. If the length of the tokenized question is less than $N_q$, a predetermined token length, the remaining portion of the question is padded with BERT's \verb|[mask]| token. Otherwise, it is truncated. This process, known as *query augmentation*, allows BERT to re-weight existing terms or expand the query, and it is pivotal to ColBERT's performance. The generated embeddings are then passed through a linear layer to reduce the output dimensions to a fixed size $m$, which is smaller than the original dimensions of BERT. The output is subsequently normalized to ensure that the L2 norm of each result equals one.

For each passage $p$, $E_P\left(\cdot\right)$ is employed for encoding. Similar to the question encoding process, $p$ is segmented into its $p_1, \ldots, p_{T_d}$ Wordpiece tokens. The special token \verb|[D]| indicates a passage. Short passages are not padded with a \verb|[mask]| token. After the classical BERT output, a similar post-processing step is applied to the encoded passages, and all embeddings corresponding to punctuation are filtered out.

\begin{equation}
    \mathbf{E}_{q}:= Normalize(CNN(BERT("[Q] q_0 q_1 \dots q_T [mask] \dots [mask]")))
\end{equation}

\begin{equation}
    \mathbf{E}_{p}:= Filter(Normalize(CNN(BERT("[D]p_0 p_1 \dots p_T"))))
\end{equation}

The late interaction mechanism applied to the encodings involves computing the maximum similarity, which utilizes cosine similarity through dot products. This is made possible by the earlier normalization applied to the embeddings:

\begin{equation}
    \mathbf{s}_{q, p}^{C o l B E R T}=\sum_{I\in\left[\left|\mathbf{E}_q\right|\right]}\max _{j\in\left[\left|\mathbf{E}_d\right|\right]} \mathbf{E}_{q, i} \cdot \mathbf{E}_{p, j}^{\top}
\end{equation}

The interaction mechanism has no trainable parameters. ColBERT is differentiable end-to-end. During training, for example, with a triple $(q, p^+, p^-)$, ColBERT independently produces a score for each passage and is subsequently optimized pairwise using softmax cross-entropy loss over the scores of $p^+$ and $p^-$ \cite{khattab_colbert_2020}.

Another type of Retriever is the \textbf{Iterative Retriever}. Iterative Retrievers are necessary when dealing with questions that are more complex than simple factoid questions, which can be answered by identifying the right passage in the knowledge source. An example is the HotpotQA dataset \cite{yang_hotpotqa_2018}, designed specifically for multi-hop questions. The fundamental concept here is that such questions cannot be answered with just one precise piece of evidence. They require multiple passages from different documents at the very least. Iterative Retrievers encompass three stages in the pipeline: (1) document retrieval, (2) query reformulation, and (3) retrieval stopping.

An example is BEAM, currently holding the title of the highest-performing\footnote{Status as of September 23, 2023, according to https://paperswithcode.com and the authors of \cite{zhang_beam_2023}}, \gls{qa}-System across multi-hop \gls{qa} datasets such as HotpotQA \cite{zhang_beam_2023}. The document retrieval component can take the form of any retrieval model, including options like ColBERT, BM25, or \gls{dpr}. In the case of BEAM, it leverages an Interaction-Based Retriever using DeBERTa. For each candidate passage $p_c$, BEAM calculates a relevance score concerning this passage within the context of all previously identified relevant passages $p_r$ and the question $q$, using the embeddings of the \verb|[CLS]| tokens \cite{he_deberta_2020}. The second step, query reformulation, can be executed explicitly or implicitly, meaning it can either be expressed in natural language or as a dense embedding. The advantage of using natural language lies in its interpretability; while employing dense embeddings operates within a semantic space and does not lack vocabulary interpretability \cite{zhu_retrieving_2021}. BEAM adopts a natural language-based approach. Specifically, after each hop, it appends the newly identified passage to the previously identified ones and feeds this information into DeBERTa:

\begin{equation}
    \text{s}_{q, p}^{BEAM} = \text{Classifier}(\text{DeBERTa}("[CLS] q \: p_{r_1} \: \ldots \: p_{r_i} \: ")) \quad | \quad p_{c} \in P
\end{equation}

The nature of query reformulation depends on the type of retriever in use. Lastly retrieval stopping poses its own set of challenges. A common approach involves setting either a fixed number of hops or a maximum limit on the retrieved documents. Alternatively, some methods introduce a new token, such as \verb|[EOE]| (End-of-Evidence), to signal the end of retrieval \cite{zhu_retrieving_2021}. BEAM, for example, employs a fixed number of hops, specifically 2, as determined through empirical evaluation.

The task of \textbf{Document Post-Processing} is to reduce the number of passages forwarded to the Reader, aiming to eliminate irrelevant ones. Traditional Retrievers, like Sparse Retrievers, often require a Document Post-Processor. However, Dense Retrievers often incorporate ranking and retrieval simultaneously, rendering this module unnecessary \cite{zhu_retrieving_2021}. Nevertheless, it remains possible to construct multi-stage Retrievers. This can be achieved by using a simpler Dense Retriever for pre-filtering passages and subsequently applying a more accurate one \cite{liu_dense_2021}.

% Commit #4

\subsection{Reader Approaches}
\label{subsec:qa_reader}

Readers originally emerged from the field of \gls{mrc}, where the objective is to extract an answer from a given context. A well-known example is the SQuAD \cite{rajpurkar_squad_2016} dataset, which was mentioned in Section \ref{subsec:qa_basics}. However, unlike the original \gls{mrc} task, a Reader in a Retrieval-Reader-System must process multiple passages to determine the relevant information needed to answer a given question \cite{zhu_retrieving_2021}. 

Modern readers rely on \gls{prlm}s since they establish new baselines on well-known datasets \cite{luo_choose_2022}. In general, there are two types of Readers that use \gls{prlm}s: \textbf{Extractive Readers} and \textbf{Generative Readers} \cite{jurafsky_speech_2023,zhu_retrieving_2021,luo_choose_2022}.

In general, an \textbf{Extractive Reader} employs an encoder to identify the token sequence span that is relevant for answering a question. These encoders can be any autoencoder models, such as BERT \cite{devlin_bert_2019}, DeBERTa \cite{he_deberta_2020}, or RoBERTa \cite{liu_roberta_2019}. Luo et al. \cite{luo_choose_2022} even utilized the encoder components of established encoder-decoder models like T5 \cite{raffel_exploring_2023} and BART \cite{lewis_bart_2019}. They demonstrated that, after fine-tuning, these models can outperform encoder-only models on certain tasks.

Figure \ref{fig:extractive_reader} illustrates the span labeling process performed by the extractive reader. The question tokens $q_1, \ldots, q_n$ and the passage tokens $p_1, \ldots, p_m$ are input into the encoder, separated by a \verb|[SEP]| token. The encoder learns two new embeddings, $S$ and $E$, which represent span-start and -end tokens, respectively. To obtain the span start probability for an output token $p_i^{\prime}$, the dot product between the output token and $S$ is computed and then normalized by a softmax function over all output tokens. The process is similar for the span-end token. The score of a span from position $i$ to $j$ is calculated as $S * p_{i}^{\prime} + E * p_{j}^{\prime}$. The span with the highest score, where $j \geq i$, is selected as the answer span. If the total length of tokens in $q$ and $p$ exceeds the maximum input length of the encoder, the passage is split into multiple segments, and the process is repeated for each segment \cite{jurafsky_speech_2023,luo_choose_2022}.

\begin{figure}
    \centering
    \includegraphics[width=\textwidth]{Grafiken/Extractive_Reader.png}
    \caption{Adjusted Graphic of the Extractive Reader by Jurafsky et al. \cite{jurafsky_speech_2023}}
    \label{fig:extractive_reader}
\end{figure}

The \textbf{Generative Reader} operates straightforwardly when familiar with a \gls{s2s} encoder-decoder model. Given a dataset containing $(q,p,a)$ tuples, the encoder takes $q$ and $p$ as input and outputs the contextual representation $h$. Then, it is the decoder's task to generate a token sequence based on $h$ and attention. The training objective can be described as minimizing the following loss function:

\begin{equation}
    \mathcal{L}_{\mathrm{Gen}}=\sum_{i=1}^K \log P\left(a_i \mid h, a_{: i}\right)
\end{equation}

Here, $K$ represents the length of tokens in $a$, $a_i$ is the $i^{th}$ token in $a$, and $a_0$ is a special beginning of sequence token. In cases where the answer is not contained within the passages, the \verb|[CLS]| token indicates this situation \cite{luo_choose_2022,zhu_retrieving_2021}.

Latest research projects like Visconde \cite{pereira_visconde_2022} even employ \gls{llm} as Generative Readers. The performance and usability of these models remain active topics of research.

Luo et al. conducted the first survey comparing state-of-the-art Extractive and Generative Readers \cite{luo_choose_2022}. They discovered that \enquote{on average, extractive readers perform better than generative ones} \cite{luo_choose_2022}, except in cases involving long context passages, where generative approaches outperform the extractive ones.

\textbf{RAG} can be seen as a generative reader, but with a much more capable \gls{nn} as the reader, specifically the idea is that the reader itself is a \gls{llm} with implicit knowledge encoded in its parameters, which it uses to generate an answer, the retrieved passages intentionally function as support in order to guide the reader and reduce risk of hallucination. 

\begin{figure}
    \centering
    \includegraphics[width=\textwidth]{Grafiken/RAG-Figure1.png}
    \caption{Overview of RAG by Lewis et al. \cite{lewis_retrieval-augmented_2021}}
    \label{fig:rag}
\end{figure}

Figure \ref{fig:rag} is taken from the original paper by Lewis et al. \cite{lewis_retrieval-augmented_2021} and displays the general approach of \gls{rag}. The original idea of \gls{rag} is to have an end-to-end backpropagation in order to train the retriever and reader (generator) at once and on the same data, not separate as in most Retriever-Reader-Systems. The used retriever in the original \gls{rag} is a \gls{dpr}. Other retrievers can be used, as this is just a decision to make as the generator does not directly depend on the type of retriever. More interestingly is the kind of implementation of the generator. \gls{rag} implements a \textit{sequence-based} generator, while future work, such as \gls{fid} \cite{izacard_leveraging_2021} use an \textit{attention-based} generator. The sequence-based generator works the following way: Given an arbitrary encoder-decoder $p_{\theta}(y_i | q, p, y_{1:i-1})$, the query $q$, the $k$-relevant passages $p$, and the previously generated tokens $y_{1:i-1}$, the generator computes the probability distribution over the next token $y_i$. $q$ and $p$ where simply concatenated.

Further \gls{rag} generators are attention-based like \gls{fid} \cite{izacard_leveraging_2021}. Here the encoder and decoder of the generator are slightly decoupled as to the classic \gls{rag}. Given a question $q$, the retriever retrieves the top-$k$ passages $p$. The encoder encodes every single passage in a question, title, passage triple $(q, t, p)$. The encodings of multiple passages are afterward concatenated and passed into the encoder-decoder attention of the decoder. An illustration can be found in Figure \ref{fig:fid} This allows for the combination of multiple passages, so there is no input token limitation as for the classical \gls{rag}. Also, experiments by Izacard et al. showed, that the performance improves over multiple tasks, as multi-passage relations can easily be resolved by the decoder. The latest work of Izacard et al. is ATLAS, which set new state-of-the-art benchmarks on multiple evaluation tasks \cite{izacard_atlas_2022}. ATLAS extends on the idea of \gls{fid}.   

\begin{figure}
    \centering
    \includegraphics[width=\textwidth]{Grafiken/fid.png}
    \caption{Overview of FiD by Izacard et al. \cite{izacard_leveraging_2021}}
    \label{fig:fid}
\end{figure}

Still overall \gls{rag} approaches, the main idea is to have a fully end-to-end backpropagation during training or fine-tuning of the systems and a \gls{llm} whose generation is supported by passages.

The \textbf{Answer Post-Processor} is similar to the Document Post-Processor, serving as an optional component. Its primary task is to provide support for multi-hop complex questions, helping determine the final answer from a set of answers extracted by the reader component \cite{zhu_retrieving_2021}. Depending on the implementation of the Reader, this component may become obsolete.

% Commit #5

\subsection{Limitations}
\label{subsec:qa_limitations}

The evaluation metrics for \gls{ir} systems will be discussed in detail in Chapter \ref{chap:eval}. In general, selecting the components and models for an \gls{ir} system always involves a trade-off between accuracy, memory consumption, and inference speed \cite{zhang_survey_2023}.

Accuracy is primarily determined by the chosen Retriever-Reader-System. Sparse retrievers often lack a certain degree of semantic understanding, resulting in less accurate retrieved passages. In contrast, Dense Retrievers can achieve higher levels of accuracy but require thorough evaluation and training for the desired use case. Thakur et al. demonstrated that high-accuracy Dense Retrievers like \gls{dpr} can underperform in zero-shot scenarios compared to BM25 by -47.7\% \cite{thakur_beir_2021}. This highlights another crucial limitation of all \gls{nn}-based retrievers and readers: training. BM25 is, by nature, an unsupervised model for \gls{ir}, while common approaches for Dense Retrieval usually belong to the group of supervised models. These models heavily depend on their training data, whereas a Sparse Retriever like BM25 can be used without any training. According to experiments conducted by Thakur et al. \cite{thakur_beir_2021}, the best-performing out-of-distribution standalone Retrievers are Representation-Interaction Retrievers like ColBERT. Nevertheless, there exist enhanced approaches like \textit{Mixture-of-Experts}/\textit{Hybrid-Search}  or the multi-stage retrievers like \textit{BM25+CE} (will be discussed in Section \ref{subsec:retriever}), which are an implementation of the mentioned \textit{Document Post-Processing}. These approaches can outperform standalone Retrievers.

Constructing a training dataset for a \gls{qa} task can be a tedious process, as these datasets must consist of tuples in the form of $(\text{question, context, answer})$, which is not always feasible. One established research direction to address this issue is \gls{qg} \cite{serban_generating_2016}. In \gls{qg}, a \gls{s2s} model is employed to generate questions and answers based on a given passage.

Zhang et al. provide an example of \gls{dpr} applied to the Natural Questions dataset in their survey on efficient \gls{odqa} \cite{zhang_survey_2023}. The total processing time for a query is 0.91 seconds\footnote{It's important to mention that \gls{dpr} is a Representation-based Retriever, which allows offline storage of passage embeddings. The result was obtained using an Nvidia GeForce Rtx 2080 Ti GPU, averaged over 1000 examples}. This time is divided into 74.79\% for evidence search and 23.95\% for reading. The total memory cost is 79.32GB, with the index occupying 81.95\%, the raw corpus 16.39\%, and the model 1.66\%. Approaches to optimize this may include:

\begin{enumerate}
    \item Reducing Processing Time: (1) Accelerating Evidence Search, (2) Accelerating Reading
    \item Reducing Memory Cost: (1) Reducing Index Size, (2) Reducing Corpus Size, (3) Reducing Model Size
    \item One-stage Frameworks: (1) Directly Generating Answers, (2) Directly Retrieving Answers
\end{enumerate}

Techniques used in this context may include:

\begin{enumerate}
    \item Data-based: (1) Passage Filtering, (2) Dimension Reduction, (3) Product Quantization
    \item Model-based: (1) Model Pruning, (2) Knowledge Distillation, (3) Knowledge Source 
\end{enumerate}

A common technique, which is used in nearly every experimental setup for \gls{qa}-Systems, is FAISS \cite{johnson_billion-scale_2017}, a GPU-optimized implementation of the exact $k$-means clustering algorithm.

For a detailed overview of approaches towards more efficient \gls{odqa} systems, please refer to the comprehensive survey by Zhang et al. \cite{zhang_survey_2023}.

% Commit #6

\section{Conversational Question Answering}
\label{sec:cqa}

The differentiation of \gls{convqa} towards \gls{qa} will be discussed in Section \ref{subsec:cqa_basics}. This section also introduces the fundamental concepts of \gls{convqa} which are necessary to understand challenges and necessary components compared to a regular \gls{qa}-System. Section \ref{subsec:cqa_contextual_query_understanding} will cover approaches toward the concept of query expansion. Section \ref{subsec:cqa_initiative} will clampse on the concept of initiative and further approaches towards a Conversation Manager.

 Most crucial for the research of this thesis are the concepts introduced in Section \ref{subsec:cqa_basics}. Especially how a \gls{convqa} distinguishes from a regular \gls{qa}-System and the core concepts of a conversation. The approaches introduced in Section \ref{subsec:cqa_contextual_query_understanding} are good to know, but the thesis will later focus mostly on query rewriting. The other approaches are necessary for alternative system designs which will not be discussed. Section \ref{subsec:cqa_initiative} only scratches the surface of the research field of \textit{Initiative}. This is a highly complex field, with ongoing research and won't be covered in this thesis.

\subsection{Basics}
\label{subsec:cqa_basics}

Core concepts in the field of \gls{odcqa} towards a conversation in terms of \gls{convqa} are: \textit{Turns}, \textit{Hisotry}, \textit{Memory}, \textit{Session} and \textit{Dialog Features} and \textit{Dialog State} \cite{zamani_conversational_2023}. It's important to mention, that in other subdomains/-tasks of \gls{cis} more concepts are introduced, such as \textit{State}, those are not necessary or applicable for \gls{odcqa} \cite{zaib_conversational_2021}.

Figure \ref{fig:conversation_explain} shows the core concepts based on a chat. Firstly, a \textbf{Turn} is a question-response pair. Whereas a conversation usually consists of multiple turns (multi-turn). \gls{coqa} is a dataset published in 2019 by researchers at Stanford in order to extend the known \gls{qa} dataset SQuAD towards a conversational dataset, whereas on average one conversation session consists of 15 turns \cite{reddy_coqa_2018}. Multi-turns are the main distinguisher between the in Section \ref{sec:qa} introduced task, to a \gls{convqa} task. In a multi-turn scenario natural language phenomena like \textit{coreference} (multiple expressions referring to the same thing) or \textit{ellipsis} (omitting words or topics implied by the context) can occur. While in regular \gls{qa} the System will only be challenged with single-turn scenarios, so only one question, which needs an answer, in \gls{convqa} the systems have to face multi-turn scenarios, where a user might also ask followup question or in general multiple questions after each other. A \textbf{Hisotry} is consequently a set of turns that belong to one conversation session. A \textbf{Session} is a in it completed conversation. Lastly, the \textbf{Memory} is an abstract entity in which the \gls{convqa}-System stores knowledge related to a history, session or even user in general \cite{zamani_conversational_2023,gao_neural_2022}. The extent of this depends on the implementation of memory in the \gls{convqa} pipeline, which will be discussed in Section \ref{subsec:cqa_contextual_query_understanding}.

\textbf{Dialog Features} need to be assessed extra to the other mentioned concepts. While the other concepts tackle the frame of the conversation, the dialog feature evaluates the user questions themself. Possible dialog features may include \textit{drilling-down} questions, \textit{topic-shift}, \textit{clarification} or \textit{definition}. Different dialog features call for different responses by the system \cite{gupta_conversational_2020}. The \textbf{Dialog State} has to be assessed similarly. The dialog state represents the relation between turns. In cases of pre-defined domains methods like state slots are used, e.g. \verb |Date _, Location _, Artist _| have to be filled during the conversation in order to retrieve the correct information from the \gls{kb} \cite{rastogi_schema-guided_2020}. Open-Domain \gls{convqa} usually don't track the state \textit{explicitly}, but rather track it \textit{implicitly} via the type of implementation of the \textit{Contextual Query Understanding} unit.


\begin{figure}[h]
    \centering
    \includegraphics[width=\textwidth]{Grafiken/Conversation_Explain.png}
    \caption{Concepts of a Conversation in regards to a CIS}
    \label{fig:conversation_explain}
\end{figure}

Regarding the System architecture of a \gls{convqa} there is no one fits them all solution at the moment, but Gao et al. \cite{gao_neural_2022} presented a modern system architecture, which represents commonly used approaches and their corresponding components in a general fashion. This general architecture can be observed in Figure \ref{fig:convqa_system_architecture}. Modern \gls{convqa} systems are closely related to \gls{qa} systems, but lag certain generalizing components in order to be full \gls{cis} systems \cite{zamani_conversational_2023}.

\begin{figure}
    \centering
    \includegraphics[width=\textwidth]{Grafiken/System_Architecture_ConQA.png}
    \caption{General System Architecture of a Conv QA System by Gao et al. \cite{gao_neural_2022}}
    \label{fig:convqa_system_architecture}
\end{figure}

Similar to the retriever-reader architecture introduced in Section \ref{subsec:qa_architectures}, a \gls{convqa} is made up of those two components as well. The retriever has to understand the context, so the history of a conversation and retrieve based on that the most relevant documents. The reader on the other hand is closely related to the reader of a classic retriever-reader architecture \cite{zamani_conversational_2023,gao_neural_2022}. Some implementations even feed into the reader component the context in order to rank the retrieved passages better and generate a more accurate answer \cite{owoicho_trec_2022}.

\subsection{Contextual Query Understanding}
\label{subsec:cqa_contextual_query_understanding}

There are two main distinguishing approaches towards history implementation. The first is a simple heuristic of using the last-$k$ turns for \textbf{Query Expansion}, \textbf{Query Rewriting} or \textbf{Conversational Retrievers}. The second is to extract the important parts of the history in regards to a question and use them for Query Expansion or Rewriting \cite{gao_neural_2022}.

A good example to explain the second approach towards extracting important parts of the history $H = {(q_1, a_1), \dots , (q_i, a_i)}$ given a new question $q_{i+1}$ is \gls{quretec}\cite{voskarides_query_2020}. \gls{quretec} consists of two components essentially: one BERT-based model and a trainable classification layer. The $H$ is being passed through the BERT model, whereas the following structure of concatination is being used: 

\begin{equation}
    \text{BERT}([CLS],H,[SEP],q_{i+1}) 
\end{equation}

On every first sub-token of a term of the $H$ the term classification layer is applied, which is a network consisting of a dropout layer, a linear layer and a sigmoid function. The term classification layer predicts a label between $0-1$ indicating its importance for answering the new question $q_i$. This leads to a set of terms $I$ which need to be incorporated into the retrieval \cite{voskarides_query_2020}. This is generally also known as \textbf{Query Expansion}, whereas we add terms to a given query for retrieval. Next to this supervised, trained approach, there are also implementations that work unsupervised like Historical Query Expansion (HQExp) \cite{yang_query_2019}, which was one of the best-performing models in the TREC CAsT 2019 \cite{dalton_trec_2020}.

Modern neural approaches more often implement a \textbf{Query Rewriting} module which is built on top of \gls{s2s}-models to rewrite a query $q_{i+1}$ given a history $H$ in order to use the generated new query for retrieval using an established \gls{qa} retriever \cite{owoicho_trec_2022}. The main advantage of this approach is the absence of the need for large supervised datasets as for Conversational Retrievers \cite{dai_dialog_2022}. One of the top performing models in the TREC CAsT 2022 was HEATWAVE by a Team of the University of Cambridge England \cite{liusie_university_nodate}. HEATWAVE utilized a query rewriter and a classical lexical BM25 retriever in combination with a BERT-based re-ranker. The rewriter uses a T5-based Transformer model and gives as input $ctx-n-m$, where $n$ refers to the last $n$-many user utterances and $m$ to the $m$-many system responses. In general, the task can be simply broken down into the following:

\begin{equation}
    q_{rewritten} = \text{Rewriter}(ctx-n-m)
\end{equation}

For training of this model, they used among others the canard dataset \cite{elgohary_can_2019} a manually annotated version of the QuAC dataset, specifically for the task of query rewriting given a conversation history $H$.

State-of-the-art research utilizes more and more \gls{llm} for the task of Query Rewriting, as they can handle long context histories $H$ and are in general strong zero- or few-shot models \cite{mao_large_2023}. This is also the main approach frameworks like \textit{Langchain} \cite{noauthor_question_nodate} or \textit{ChatGPT by OpenAI} \cite{noauthor_chatgpt_2023} use. 


Lastly the approach of \textbf{Conversational Retrievers} exists. Those use compared to classical \gls{qa} retrievers not a pair of $(q,p)$ in order to calculate a similarity $sim(q,p)$ between the question $q$ and the passage $p$, but use conversational interactions like $(q_1, a_1, \dots, q_i, a_i, q_{i+1}, p)$, in short $(H, q_{i+1}, p)$, so combining a conversation history $H$ with a new question $q_{i+1}$ and the relevant passage $p$ to answer this question given the history $H$ \cite{gao_neural_2022,dai_dialog_2022}. High-performing zero-shot or subdomain-adapted Conversational Retrievers do not exist, as it is extremely time-consuming to create a dataset for this type of Retriever. To close this gap, researchers proposed sufficient data augmentation techniques to generate those datasets, given a document. One example is the work of Dai et al. \cite{dai_dialog_2022} which introduced the technique of \enquote{Dialog Inpainting} \cite{dai_dialog_2022}.  

\subsection{Initiative}
\label{subsec:cqa_initiative}

Most modern human-computer interactions follow a one-sided initiative model, where either the user- or the system-initiative is given. In mixed-initiative scenarios of \gls{convqa} the system can take initiative without explicit commands of the user. Examples of initiative are: \textit{Topic Shifts}, \textit{Clarification Questions} or \textit{Question Recommendations} \cite{zamani_conversational_2023}. In this thesis, we will focus on \textit{Clarification Questions} only.

Asking \textit{Clarification Questions} is the task of identifying ambiguity in a user's search request and resolving it by posing a question with the intent to eliminate ambiguity. A common taxonomy to use for the types of ambiguous questions includes: 1) questions where the focus is ambiguous, and 2) questions with several distinct possible answers \cite{larsson_issue-based_2002}. Several studies have been conducted to understand user behavior in relation to clarification in search. Tavakoli et al. \cite{tavakoli_analyzing_2021} found that users are more likely to engage when a \textit{Clarification Question} aims at clarifying ambiguous information instead of seeking confirmation or similar. Zamani et al. \cite{zamani_analyzing_2020}, based on search-engine log analysis, found that users are more likely to engage with a \textit{Clarification Question} if their own question has high ambiguity and, therefore, multiple possible resolutions or when there is a dominant assumed search intent.

To resolve ambiguity in the context of \gls{convqa}, the most modern solutions follow a two-step approach, where, in the first step, the system identifies the ambiguity, and in the second step, the system generates a \textit{Clarification Question} to resolve the ambiguity \cite{kuhn_clam_2023, guo_abg-coqa_nodate}. The exact implementation may differ, whereas Guo et al. \cite{guo_abg-coqa_nodate} developed a \gls{s2s}-model to predict the ambiguity of a question given context, Kuhn et al. \cite{kuhn_clam_2023} used a chain-of-thought-like approach, where a \gls{llm} performs both steps sequentially. In general, it has to be said that, in the context of \gls{odqa}, the niche of \textit{Clarification Questions} is not well-researched yet, and there is no established benchmark dataset.

\section{Efficient Large Language Models}
\label{sec:efficient_llm}

With the increasing size of large language models (\gls{llm}), \gls{llama} 2 offers models ranging from 7 billion to 70 billion parameters \cite{touvron_llama_2023}. Even these models are considered relatively small compared to the largest models like PaLM 2 \cite{anil_palm_2023} with 340 billion parameters. The challenge arises when running such models in scenarios with limited computational resources, especially on smaller domains or tasks. This challenge is particularly relevant to the task presented in this thesis, which involves building a \gls{convqa} system for a custom set of documents.

While several surveys \cite{ling_domain_2023, zhao_survey_2023} have explored the topic of efficient \gls{llm} usage in resource-constrained systems, Treviso et al. \cite{treviso_efficient_2023} present the most comprehensive taxonomy of methods and approaches in this context. Figure \ref{fig:llm_taxonomy} provides a high-level overview of the stages at which efficiency-improving methods can be implemented in \gls{llm}s. Given the specific focus of this thesis, not all stages will be discussed in detail. For more comprehensive insights, please refer to the original survey by Treviso et al. \cite{treviso_efficient_2023}.

\begin{figure}
    \centering
    \includegraphics[width=\textwidth]{Grafiken/Efficient_Survey_Steps.png}
    \caption{Adapted Stages of Efficiency Improvement for LLMs by Treviso et al. \cite{treviso_efficient_2023}}
    \label{fig:llm_taxonomy}
\end{figure}

Section \ref{subsec:llm_fine_tuning} will explore possibilities to enhance efficiency during the fine-tuning process, while Section \ref{subsec:llm_compression} will delve into the topic of model compression, which is applicable to the \textit{Inference} step in Figure \ref{fig:llm_taxonomy}.

 This thesis itself won't delve into fine-tuning (the content of Section \ref{subsec:llm_fine_tuning}) or model compression (the content of Section \ref{subsec:llm_compression}) itself. Still, concepts like multi-task learning are necessary to understand the capabilities and where those are coming from for modern \gls{llm}s, and the technique of \textit{Prompting} will be used in Chapter \ref{chap:eval}. Also, in Chapter \ref{chap:eval}, quantized models will be used to evaluate the performance of the \gls{convqa}-System, out of the necessity to run \gls{llm}s on the available hardware resources. Nevertheless, the following section can be seen as an excursion to the main topic of this thesis for real-world implementations facing resource-constrained systems.


% \begin{figure}
%     \centering
%     \includegraphics[width=\textwidth]{Grafiken/Efficient_Survey_Taxonomy.png}
%     \caption{Adjusted Taxonomy of efficiency improvement for \gls{llm} by Treviso et al. \cite{treviso_efficient_2023}}
%     \label{fig:llm_taxonomy_2}
% \end{figure}

\subsection{Fine-Tuning}
\label{subsec:llm_fine_tuning}

Hu et al. \cite{hu_lora_nodate} demonstrated the significant benefits of fine-tuning GPT-3 for few-shot applications, highlighting the remarkable improvements fine-tuning can achieve. This is further supported by the experiments conducted by Chung et al. \cite{chung_scaling_2022}.

Efficient fine-tuning of \gls{llm}s can be categorized into three distinct approaches: \textit{Parameter Efficiency}, \textit{Multi-task Learning}, and \textit{Prompting}. Figure \ref{fig:llm_fine_tuning} provides an overview of these approaches along with their corresponding methods.

\begin{figure}[h]
    \centering
    \includegraphics[width=0.8\textwidth]{Grafiken/fine_tuning_approaches.png}
    \caption{Adapted Fine-Tuning Approaches for \gls{llm} by Treviso et al. \cite{treviso_efficient_2023}}
    \label{fig:llm_fine_tuning}
\end{figure}

\textbf{Parameter Efficiency} is commonly referred to as \textbf{\gls{peft}} \cite{noauthor_peft_nodate}. A notable \gls{peft} approach is \gls{lora}, developed by Hu et al. \cite{hu_lora_nodate}. \gls{lora} falls under the category of adapters, a term coined because it revolves around the concept of freezing the parameters of the \gls{llm} and fine-tuning only a small set of task-specific parameters, which can be swapped depending on the desired downstream task. Unlike some other adapter-based methods \cite{houlsby_parameter-efficient_2019}, \gls{lora} does not introduce additional inference latency due to the merging of trainable matrices with frozen weights. Moreover, \gls{lora} can be seamlessly combined with many other \gls{peft} methods.

In practice, given a pre-trained weight matrix \(W_0 \in \mathbb{R}^{d \times k}\), which is typically full-rank between layers, the update can be constrained to be a low-rank composition: \(W_0 + \triangle W = W_0 + BA\), where \(B \in \mathbb{R}^{d \times r}\), \(A \in \mathbb{R}^{r \times k}\), and the rank \(r \ll \min(d, k)\). While \(W_0\) remains frozen during training, \(A\) and \(B\) become trainable parameters. The forward pass \(h = W_0x\) can be represented as the following sum:

\begin{equation}
    h = W_0x + \triangle Wx = W_0x + BAx
\end{equation}

Figure \ref{fig:lora} illustrates the architecture and initialization during training. The parameters of \(A\) are randomly sampled using Gaussian initialization, while \(B\) is initialized to 0.

Other \gls{peft} techniques include \textbf{prompt-tuning} \cite{lester_power_2021} and \textbf{prefix-tuning} \cite{li_prefix-tuning_2021}. Both approaches are similar in the way they leverage task-specific modifications to the input to guide the model's behavior. They involve concatenating learned vectors to activations or embedding sequences, making them activation-modifying \gls{peft} methods.

A \gls{peft} approach that can be considered a hybrid between \gls{lora} and activation-modifying techniques is \textbf{$\text{(IA)}^3$} \cite{liu_few-shot_2022}. What sets $\text{(IA)}^3$ apart is its focus on \gls{llm}s designed explicitly for multi-task learning, as all existing \gls{peft} techniques significantly underperformed in experiments conducted by Liu et al. \cite{liu_few-shot_2022}. In $\text{(IA)}^3$, the model's activations are rescaled using element-wise multiplication with learned vectors, known as adaptors. Specifically, $\text{(IA)}^3$ employs three learned vectors: \(l_k \in \mathbb{R}^{d_k}\) for keys and \(l_v \in \mathbb{R}^{d_v}\) for values in self-attention and encoder-decoder attention mechanisms, as well as \(l_{ff} \in \mathbb{R}^{d_{ff}}\) for the feed-forward network. The rescaling is incorporated into the attention mechanism as follows:

\begin{equation}
    \operatorname{softmax}\left(\frac{Q\left(l_{k} \odot K^T\right)}{\sqrt{d_k}}\right)\left(l_{v} \odot V\right)
\end{equation}

For the feed-forward network, the rescaling is implemented as follows, where \(\gamma\) represents the feed-forward activation:

\begin{equation}
    (l_{ff} \odot \gamma (W_1x))W_2
\end{equation}

In summary, \textbf{$\text{(IA)}^3$} is a \gls{peft} approach specifically designed for multi-task learning, and it appears to outperform \gls{lora} in terms of the number of parameters added and training computation costs.

\textbf{Multi-task Learning} refers to the concept of fine-tuning a \gls{prlm} on various tasks to achieve better zero-shot and fine-tuning performance. One of the most prominent \gls{prlm}s trained on multiple NLP tasks is T5 \cite{raffel_exploring_2023}. Liu et al. \cite{liu_few-shot_2022} demonstrated that a generically trained T5 model (T0) can be few-shot fine-tuned with approximately 10\% of the parameters of \gls{lora}, at a lower computational cost, while achieving higher accuracy in classification tasks\footnote{Currently, there is no experiment comparing LoRa and $\text{(IA)}^3$ on QA tasks.}. This is made possible, due to using $\text{(IA)}^3$ as \gls{peft} and the multi-task pre-training of the used model. Still, this also includes the drawback of having in general a larger model.

\textbf{Prompting} refers to the general concept of presenting a task as a textual instruction to a \gls{llm} \cite{brown_language_2020}. Recent advances have even led to the development of a new sub-task and job role called \textit{Prompt Engineering} \cite{white_prompt_2023}. It's worth noting that different prompts with the same intent can yield different results, making the selection of the right prompt a challenge in itself \cite{liu_gpt_2021}. Furthermore, concepts like \gls{cot} prompts have been developed. In \gls{cot} prompts, the given example in a few-shot prompt is redesigned to mimic step-by-step reasoning and conclusions known from the way humans think, aiming to achieve higher performance in zero- and few-shot scenarios simply by adjusting the explicit natural language prompt \cite{wei_chain--thought_2023}. \gls{cot} requires retraining of the \gls{llm}.

\begin{figure}
    \centering
    \includegraphics[width=0.5\textwidth]{Grafiken/Lora.png}
    \caption{General Idea of LoRa by Hu et al. \cite{hu_lora_nodate}}
    \label{fig:lora}
\end{figure}


\subsection{Compression}
\label{subsec:llm_compression}

Compression aims to reduce the size of a model, whether it's the number of parameters while maintaining the same level of accuracy on the downstream task, or the actual storage required for the model. There are three primary approaches to this task: \textit{Pruning}, \textit{Knowledge Distillation}, and \textit{Quantization} \cite{treviso_efficient_2023, zhu_survey_2023}. Figure \ref{fig:llm_compression} provides an overview of these approaches and their corresponding methods.

\begin{figure}[h]
    \centering
    \includegraphics[width=0.8\textwidth]{Grafiken/compression_approaches.png}
    \caption{Adapted Compression Approaches for \gls{llm} by Treviso et al. \cite{treviso_efficient_2023}}
    \label{fig:llm_compression}
\end{figure}

\textbf{Pruning} can be further categorized into \textit{structured} and \textit{unstructured} pruning. Structured pruning involves removing specific patterns of weights or activations from a model, with the goal of maintaining a dense matrix representation to ensure compatibility with existing implementations and hardware. One notable example of a structured pruner for \gls{llm}s is LLM-Pruner \cite{ma_llm-pruner_2023}. On the other hand, unstructured pruning entails removing individual weights or activations from a model, resulting in a sparse matrix representation. This approach may require specialized hardware or software implementations to efficiently compute and achieve speed improvements of 1.5$\times$ to 2.16$\times$, while reducing up to 60\% of the parameters \cite{frantar_sparsegpt_2023}. Examples of engines designed specifically for unstructured pruning include NVIDIA's CUTLASS library for GPUs \cite{frantar_sparsegpt_2023} and DeepSparse \cite{noauthor_deepsparse_2023} for CPUs.

\textbf{Knowledge Distillation} is an approach that involves using a generally well-performing \gls{llm} as a teacher to instruct a significantly smaller student model \cite{hinton_distilling_2015}. Zhu et al. distinguish between \textit{White-box} and \textit{Black-box} knowledge distillation. In the former, the student has full access to the teacher's parameters, while in the latter, only the teacher's predictions are accessible to the student \cite{zhu_survey_2023}. An example of white-box knowledge distillation is MiniLLM \cite{gu_knowledge_2023}, where the distribution of the final layer's outputs for both the teacher and the student, given a prompt, is compared using the Kullback-Leibler divergence. This comparison is used in a loss function for backpropagation in the student model.

Black-box approaches are more commonly used in knowledge distillation. In these cases, a \gls{llm} is employed to either directly provide its predictions based on a prompt \cite{huang_-context_2022}, offer assisting explanations \cite{li_explanations_2022}, or sort the training data by difficulty and artificially generate more data points \cite{jiang_lion_2023}, among other techniques. For a comprehensive overview, please refer to Section 2.2 \textit{Knowledge Distillation} in Zhu et al.'s survey \cite{zhu_survey_2023}. Experiments with different distillation approaches have shown that distillation has its limitations, and for specific downstream tasks, fine-tuning can outperform knowledge distillation \cite{zhu_teach_nodate}.

\textbf{Quantization} is an approach that involves reducing the datatype representation of weights or activations, which are typically floating-point numbers, to smaller representations in terms of bits, such as 8-bit integers or even smaller discrete formats \cite{gholami_survey_2021}. Generally, there is a distinction between \textit{Quantization-aware Training} and \textit{Post-Training Quantization}. The names are self-explanatory; the former involves applying and adjusting quantization during the training process (either pre-training or fine-tuning) \cite{liu_llm-qat_2023}, while the latter pertains to quantization after the training is completed \cite{frantar_gptq_2023}. In both cases, numerous approaches and methods exist, applying different paradigms and quantization techniques, including decisions regarding which parameters to quantize, structured vs. unstructured quantization, quantization strength, and many others. It's not possible and necessary to discuss all of these here, but for a comprehensive overview, please refer to Section 2.3 \textit{Quantization} in Zhu et al.'s survey \cite{zhu_survey_2023}.

The most prominent example of Post-Training Quantization is GPTQ, which is the only ready-to-use implementation available in the Huggingface Transformer Library \cite{noauthor_quantize_nodate}. GPTQ was the first method to achieve high compression for \gls{llm}s with over 175 billion parameters while maintaining high accuracy compared to prior state-of-the-art algorithms. Specifically, with a 4-bit quantization of the weights, GPTQ achieved approximately 5$\times$ compression for BLOOM-176B and OPT-175B, two openly available \gls{llm}s, while experiencing only a $\leq$ 0.25 decrease in perplexity compared to the original model. Therefore, the following section will explain the Post-Training Quantization approach of GPTQ in detail.

\textbf{GPTQ} builds upon \gls{obq}, the previous work by Frantar et al. on Post-Training Quantization \cite{frantar_optimal_2023}. With GPTQ, their objective was to reduce the runtime complexity of \gls{obq}, which is $O(d_{row} * d_{col}^{3})$, making it compatible with \gls{llm}s containing billions of parameters. The central idea behind GPTQ is a layer-wise optimization approach. The aim is to discover quantized weights $\widehat{\mathbf{W}}$ that minimize the squared error compared to the full-precision layer $\mathbf{W}$ output, using a given set of input data points $\mathbf{X}$:

\begin{equation}
    \operatorname{argmin}_{\widehat{\mathbf{W}}}\|\mathbf{W} \mathbf{X}-\widehat{\mathbf{W}} \mathbf{X}\|_2^2
\end{equation}

In \gls{obq}, we denote the next weight to be quantized as $w_q$. We define the function $\text{quant}(w)$, which rounds a weight $w$ to the nearest value on the quantization grid.

\begin{equation}
    w_q=\operatorname{argmin}_{w_q} \frac{\left(\text { quant }\left(w_q\right)-w_q\right)^2}{\left[\mathbf{H}_F^{-1}\right]_{q q}}
\end{equation}

In the context of \textbf{GPTQ}, a column of weights is always updated simultaneously. Therefore, $\text{quant}(W_{:,j})$ refers to the following:

\begin{equation}
    \text{quant}(W_{:,j}) := \operatorname \forall w_q \in W_{:,j} 
\end{equation}

The Hessian matrix $\mathbf{H_F} = 2X_FX_F^T$ is utilized for both weight updates and quantization error calculations. Once all columns within a block $B$ are quantized, the weight update is computed as follows, where $Q$ represents the set of indices corresponding to quantized weights:

\begin{equation}
    \boldsymbol{\delta}_F = -\left(\mathbf{w}_Q-\text { quant }\left(\mathbf{w}_Q\right)\right)\left(\left[\mathbf{H}_F^{-1}\right]_{Q Q}\right)^{-1}\left(\mathbf{H}_F^{-1}\right)_{:, Q} \\
    \label{eq:weight_update}
\end{equation}

Furthermore, the Hessian is updated in the following manner, avoiding the need for recomputation; instead, columns corresponding to quantized weights are simply dropped from the Hessian.

\begin{equation}
    \mathbf{H}_{-Q}^{-1} = \left(\mathbf{H}^{-1} - \mathbf{H}_{:, Q}^{-1}\left[\left(\mathbf{H}^{-1}\right)_{Q Q}\right]^{-1} \mathbf{H}_{Q,:}^{-1}\right)_{-Q}.
    \label{eq:hessian_upadte}
\end{equation}

This leads to Algorithm \ref{alg:quantize_W}.

\begin{algorithm}
    \caption{Quantize $W$ given inverse Hessian $H^{-1} = (2XX^T + \lambda I)^{-1}$ and blocksize $B$ by Frantar et al. \cite{frantar_gptq_2023}}
    \label{alg:quantize_W}
    \begin{algorithmic}[1]
    \State $Q \gets \mathbf{0}_{d_{\text{row}} \times d_{\text{col}}}$ \Comment{Quantized output}
    \State $E \gets \mathbf{0}_{d_{\text{row}} \times B}$ \Comment{Block quantization errors}
    \State $H^{-1} \gets \text{Cholesky}(H^{-1})^T$ \Comment{Hessian inverse information}
    
    \For{$i \gets 0, B, 2B, \ldots$}
        \For{$j \gets i, \ldots, i + B - 1$}
            \State $Q_{:,\,j} \gets \text{quant}(W_{:,\,j})$ \Comment{Quantize column}
            \State $E_{:,\,j-i} \gets \frac{W_{:,\,j} - Q_{:,\,j}}{[H^{-1}]_{j,j}}$ \Comment{Quantization error}
            \State $W_{:,\,j:(i+B)} \gets W_{:,\,j:(i+B)} - E_{:,\,j-i} \cdot H^{-1}_{j,j:(i+B)}$ \Comment{Update weights in block}
        \EndFor
        \State $W_{:,(i+B):} \gets W_{:,(i+B):} - E \cdot H^{-1}_{i:(i+B),(i+B):}$ \Comment{Update all remaining weights}
    \EndFor
    \end{algorithmic}
\end{algorithm}

To enable GPTQ to be applicable to \gls{llm}s with billions of parameters, the authors have introduced three key optimizations:

\begin{enumerate}
    \item \textit{Arbitrary Order:} In the case of large models, the order in which weights are quantized becomes irrelevant. Therefore, GPTQ updates all weights in the same order for all rows. This means that the set of unquantized weights, denoted as $F$, and $H_F^{-1}$, the Cholesky Form - Inverse Layer Hessian, remain constant across all rows. This is because $H_F$ depends solely on $X_F$ and is independent of the weights. This reduction in the number of times $H$ needs to be updated simplifies the process from $d_{col} \times d_{row}$ updates to just $d_{col}$ updates.
    \item \textit{Lazy Batch-Updates:} Quantization of a column depends solely on updates to that particular column. Therefore, GPTQ employs batches of columns (with a batch size of $B = 128$). Equations \ref{eq:weight_update} and \ref{eq:hessian_upadte} can be executed after the computation of a full batch $B$. The set of indices $Q$ corresponds to the indices of quantized weights in the batch.
    \item \textit{Cholesky Reformulation:} To address numerical errors that arise from repeated application of equation \ref{eq:hessian_upadte}, a Cholesky reformulation is applied to calculate all the necessary information about $H^{-1}$ in advance. As the complete Cholesky decomposition cannot be applied, a mild damping factor is applied to the diagonal.
\end{enumerate}

Additionally, an accessible quantization package called \textbf{AutoGPTQ} has been developed, which implements the GPTQ algorithm in PyTorch \cite{william_autogptq_2023}. This package has been adopted by Hugging Face and is currently the only ready-to-use quantization technique available in the Transformers library \cite{noauthor_quantize_nodate}.

\section{Related Work}
\label{sec:related_work}

\subsection{Question Answering based on PDFs}
\label{subsec:related_work_dbqa}

\textbf{PDF Question Answering} is the task of providing answers to questions related to the content of one or multiple documents \cite{mathew_document_2021}. The field of research which actively explores this the closest is Visual Document Question Answering. It works on the development of an IR-QA system that operates on images of documents. An exemplary architecture and a general pipeline for transforming PDFs into an IR-QA system is presented by McDonald et al. \cite{mcdonald_detect_2022}. They developed their zero-shot framework around the QASPER dataset but used the original PDFs instead of extracted text via LaTeX. Moreover, readily available open-source tools like V-Doc \cite{ding_v-doc_2022} simplify the deployment and testing of datasets, models, and IR-QA systems of the Visual Document Question Answering domain.

More recently, the open-source framework \textit{Langchain} has gained tremendous attention\footnote{As of September 24, 2023, Langchain has received 63k stars on GitHub}. Langchain focuses on harnessing LLMs using chains, which are essentially prompts for an LLM that can be chained together \cite{noauthor_langchain-ailangchain_nodate}. They also provide documentation on building a QA system based on PDFs \cite{noauthor_question_nodate}. Similarly, \textit{OpenAI} offers a Retrieval Plugin for \textit{ChatGPT} \cite{noauthor_chatgpt_2023}, also an open-source repository. These QA systems adhere to the paradigms established in previous works such as \cite{karpukhin_dense_2020,ni_large_2021,neelakantan_text_2022,lewis_retrieval-augmented_2021}. Specifically, this entails:

\begin{itemize}
    \item Given a text corpus, documents can be retrieved by extracting relevant passages. Data cleaning of the corpus is optional but not necessary. Therefore, these systems employ a \textit{direct extraction} approach, especially when dealing with PDFs.
    \item Utilizing large-scale, diversely trained encoders. Representation-based Retrievers, when equipped with sufficient trainable parameters and diverse training datasets, often yield comparable results to fine-tuned, more complex retrieval models \cite{ni_large_2021,neelakantan_text_2022}.
    \item Using the LLM as a generative reader for QA, as demonstrated in the work of Izacard et al. \cite{izacard_leveraging_2021}.
\end{itemize}

Non-LLM research for \gls{qa} based on PDFs is notably scarce. In the field of ODQA, discussions regarding applicable frameworks that encompass the entire pipeline from PDFs to \gls{qa} are infrequent. Instead, the focus often revolves around constructing \gls{qa} systems using predefined and well-supervised datasets. However, there is some research that explores the feasibility of deploying high-performing \gls{qa} systems in out-of-domain scenarios, bypassing the initial stage of data preprocessing (from PDFs to passages). This research strives to outline possibilities for using a \gls{qa} system in real-world passage collections.

\noindent \textbf{Applying Dense Retrievers Out-of-Domain}: As emphasized by Thakur et al. in their \enquote{Heterogeneous Benchmark for zero-shot Evaluation of Information Retrieval Models} (BEIR) \cite{thakur_beir_2021}, dense retrievers exhibit weak out-of-domain performance. Lyu et al. \cite{farea_evaluation_2022} also demonstrate the limited generality of dense retrievers when trained in one subdomain and subsequently applied in a different one. This underscores the conclusion that there are two approaches to employing retrievers in out-of-domain scenarios: (1) fine-tuning or (2) zero-shot, but with large encoders that have been trained on diverse datasets \cite{ni_large_2021}.

The challenge with fine-tuning lies in the unavailability of labeled data, which is typically required for supervised models in the form of tuples such as $(question,\allowbreak answer,\allowbreak context)$. Several diverse approaches have been developed to address this issue. One approach employs \gls{qg} techniques, as exemplified by PROMPTAGATOR \cite{dai_promptagator_2022}, which utilizes LLMs. Another strategy involves the use of Mixture-of-Experts and meta-learning algorithms \cite{chen_improving_2021}. Some researchers have explored semi-supervised training datasets, as demonstrated by Sachan et al. \cite{sachan_questions_2023}, who developed ART, a training framework for dense retrievers that only requires questions and surpasses the standard \gls{dpr} training implementation. At the current point in time, there is no state-of-the-art approach to fine-tune a dense retriever on a small subdomain dataset.

In their study, Reddy et al. \cite{reddy_synthetic_2022} addressed the challenge of creating a \gls{qa}-System for Covid-19-related documents, where no supervised \gls{qa} dataset was available. Consequently, they conducted a comparison between the performance of zero-shot BM25 and \gls{dpr}. Their findings revealed that BM25 outperformed \gls{dpr} on the BiosQA \gls{qa} dataset, closely related to the Covid-19 domain. Throughout their experiments, they evaluated various approaches, including simple zero-shot techniques, and fine-tuning of \gls{dpr} using \gls{qg} via BART, which yielded superior results. Notably, the most effective retriever for unsupervised domain adaptation was a combination of BM25 and unsupervised fine-tuned \gls{dpr}.

Furthermore, Gururangan et al. \cite{gururangan_dont_2020} demonstrated in their experiments that fine-tuning \glspl{prlm} on domain-specific language or, even better, task-specific data led to a significant performance boost.

Gholami et al. \cite{gholami_zero-shot_2021} experimented with non-fine-tuned dense retrievers on a non-\gls{qa} dataset, specifically a collection of AWS documentations. Their results, particularly for the retrieval component, were sobering, aligning with the findings of benchmark studies by Thakur et al. \cite{thakur_beir_2021} and Lyu et al. \cite{farea_evaluation_2022}.

On the other hand, there exist reader components with a high degree of generalizability, as demonstrated by UnifiedQA-v2 \cite{khashabi_unifiedqa-v2_2022}, an extractive reader and T5 \cite{raffel_exploring_2023}, a generative reader. So the main challenge, when building a \gls{ir}-\gls{qa}-System, lies within the implementation and adaptation of the retriever component.


\subsection{Open-domain Conversational Question Answering}
\label{subsec:related_work_cqa}

\textbf{Datasets:} Notable datasets for \gls{convqa} include CoQA, TREC 2019, and QReCC, which primarily feature extractive questions \cite{reddy_coqa_2018, dalton_trec_2020, dai_dialog_2022}. While these datasets address conversation-specific challenges like coreference resolution, the absence of an adequate benchmark for \gls{convqa} is evident. Naveed et al.'s survey on \gls{llm}s \cite{naveed_comprehensive_2023} highlights various alternative datasets in this domain, largely focused on specific challenges. Recent works like Llama2 underscore the lack of a state-of-the-art benchmark, necessitating human-based evaluation as a primary approach \cite{touvron_llama_2023}.

\vspace{\baselineskip}
\noindent\textbf{Chat Fine-Tuned LLMs:} Fine-tuning \gls{llm}s on human chat-like tasks, introduced by LaMDA and later widely spread by ChatGPT, has emerged as a promising approach for conversational information retrieval \cite{thoppilan_lamda_2022, noauthor_chatgpt_2023}. These models are designed for conversational interactions and can generate human-like responses, although they may lack truthfulness, a crucial aspect for high-quality \gls{convqa} systems.

\vspace{\baselineskip}
\noindent\textbf{Truthfulness:} Addressing the issue of truthfulness, several approaches such as REALM, \gls{rag}, \gls{fid}, and WebGPT leverage external knowledge \cite{guu_realm_2020, lewis_retrieval-augmented_2021, izacard_leveraging_2021, nakano_webgpt_2022}. Among these, \gls{rag} has garnered significant research attention due to its ease of implementation and explicit knowledge, enhancing model understandability \cite{gao_retrieval-augmented_2024}.

\vspace{\baselineskip}
\noindent\textbf{Use-Case Implementation:} Research on implementing a real-world \gls{convqa} system using a specific dataset is limited. The DialDoc 2021 Shared Task presents various solutions, mainly focusing on extractive readers \cite{feng_dialdoc_2021}. However, there is a lack of research addressing the challenges, approaches, and considerations for designing such a system based on the latest advances in \gls{llm}s. The documentation of the Langchain Framework for Question Answering provides some guidens, although it lacks certain considerations \cite{noauthor_question_nodate}, such as complex retrievers or other paradigms then \gls{rag} for example \gls{fid}. Langchain mainly focuses on \gls{dpr} retrievers and explicit \gls{rag} implementations.


%%%%%%%%%%%%%%%%%%%%%%%%%%%%%%%%%%%%%%%%%%%%%%%%%%%%%%%%%%%%
\newpage
%%%%%%%%%%%%%%%%%%%%%%%%%%%%%%%%%%%%%%%%%%%%%%%%%%%%%%%%%%%%

\chapter{Open-domain QA Chatbot}
\label{chap:main}


This chapter outlines the methods and techniques employed in the development of a conversational question-answering system designed for PDFs. The chapter is structured as follows: Section \ref{sec:overview} provides an overview of the desired use case, its objectives, and constraints concerning a Conversational Question Answering System. Section \ref{sec:conrag} presents a general framework that can be utilized as a decision tree for the practical implementation of a Conversational Question Answering System for PDFs. Its subsections will highlight and discuss the components introduced within the framework.

\section{Overview and Objective}
\label{sec:overview}

The primary use case addressed in this thesis can be summarized as follows: Imagine having a collection of PDF files, and our goal is to create a chatbot capable of engaging in conversations about the knowledge within these PDFs. This chatbot should provide accurate answers to questions based on the content of the PDFs and furnish supporting evidence from these documents. Furthermore, it should enable users to have a conversational query experience, allowing them to ask follow-up questions and engage in dialogue with the chatbot based on its previous responses. Figure \ref{fig:use-case} illustrates an example of this use case.

\begin{figure}
    \centering
    \includegraphics[width=0.8\textwidth]{Grafiken/Use_Case.png}
    \caption{Overview of the Example Use-Case}
    \label{fig:use-case}
\end{figure}

Currently, to the best of my knowledge, there is no scientific paper or similar resource offering a comprehensive framework or pipeline to address this use case. This thesis aims to bridge this gap by presenting a framework and pipeline designed to tackle this specific scenario. Figure \ref{fig:overview-system-architecture} provides an overview of the system architecture. The system follows the \gls{rag} architecture, as detailed in Section \ref{subsec:qa_retrieval}, which extends the classical Retriever-Reader with a \gls{llm} as a Reader, capable of incorporating parametric knowledge. To extend \gls{rag} to a \gls{convqa}, a \gls{cqu} unit, as introduced in Section \ref{subsec:cqa_contextual_query_understanding}, is essential. This novel architecture will be termed \textbf{\gls{conrag}}. The extraction pipeline will be discussed in Section \ref{subsec:extract}, with its primary tasks being the extraction of passages from the provided set of PDFs, the creation of an index, and the optional generation of synthetic training data. The three major modules comprising the architecture, namely the \textit{Retriever}, \textit{Reader}, and \textit{\gls{cqu}}, will be elaborated in their respective sections: \ref{subsec:retriever}, \ref{subsec:reader}, and \ref{subsec:cqu}.

\begin{figure}
    \centering
    \includegraphics[width=0.8\textwidth]{Grafiken/System_Architecture.png}
    \caption{Overview of the System Architecture}
    \label{fig:overview-system-architecture}
\end{figure}

To summarize, the objectives of the QA capabilities of the system are as follows:

\begin{enumerate}
    \item Utilize \textbf{PDFs} as the primary \textbf{knowledge source}.
    \item Enable the QA-System to handle a \textbf{variety of question types}, including: \textbf{extractive}, \textbf{abstractive}, and \textbf{boolean questions}.
    \item \textbf{Provide references} to PDF snippets \textbf{as evidence to answers}.
    \item Ensure the pipeline's generalizability, allowing it to adapt to new domains or knowledge sources with \textbf{minimal or no supervision} and \textbf{small datasets}.
    \item Design the pipeline to be \textbf{feasible without the need for datacenter-grade hardware resources}, making it accessible for development on standard research hardware.
    \item \textbf{Prioritize accuracy as the primary objective}, as constraining memory consumption is indirectly covered in point (5). \textbf{Latency is not a primary concern}, as the system is not intended for real-time use and will not be optimized for that.
\end{enumerate}


Regarding the ConvQA-System, the objectives are as follows:

\begin{enumerate}
    \item Enable the ConvQA-System to \textbf{handle} the following follow-up \textbf{question types: drilling-down, clarification, topic shift} and \textbf{comparison}.
    \item Be able to take Initiative in the form of \textbf{clarifying questions}.
    \item The \textbf{memory} will be \textbf{limited to a session}.
\end{enumerate}

\section{Conversational Retrieval-Augmented Generation}
\label{sec:conrag}

As illustrated in Figure \ref{fig:overview-system-architecture}, it is logical to partition the extensive grid of possibilities into smaller, manageable components that can be explored and designed independently. Consequently, the framework will be divided into two main segments: the extraction pipeline, with its potential configurations outlined in Figure \ref{fig:extract_pipeline}, and the three major modules: Retriever, Reader, and \gls{cqu}, showcasing their possible implementations in Figure \ref{fig:all_components_conrag}.

\begin{figure}
    \centering
    \includegraphics[width=\textwidth]{Grafiken/extract_pipeline.png}
    \caption{Overview of the Extraction Pipeline Framework}
    \label{fig:extract_pipeline}
\end{figure}

The framework varies in its level of granularity, shifting between high-level concepts and precise details. This is primarily due to the fact that certain aspects of the framework are well-researched and represent state-of-the-art knowledge, while others are ongoing research and necessitate a more abstract, conceptual treatment. For instance, \textit{BM25} is mentioned specifically as the state-of-the-art Sparse Retriever within the Retriever Module, whereas \textit{Automatic Question Generation} in the Training Data Generation step of the extraction pipeline is presented as a high-level concept.

The framework strives to maintain a high level of generality, intentionally avoiding the incorporation of restrictive paradigms, except for the specified system architecture of \gls{rag} for \gls{qa}. This decision is motivated by the burgeoning breakthroughs and extensive research endeavors within the field of \gls{llm}s, as exemplified by the exceptional success of \textit{ChatGPT}. In response to the limitations of ChatGPT, including \textit{hallucination}, \textit{implicit knowledge}, and \textit{knowledge update}, interest has surged in the \gls{rag} architecture as a means to address these issues. Presently, there is no existing survey or similar resource that provides a quantitative evaluation of the ongoing business initiatives aimed at implementing RAG-based Systems. Nonetheless, both Google Cloud Services \cite{noauthor_generative_nodate} and Amazon Web Services \cite{noauthor_quickly_2023} have introduced new services that empower customers to construct \gls{rag}-based systems, with Langchain serving as the Framework for the Reader Implementation\cite{noauthor_langchain-ailangchain_nodate}. Consequently, the framework presented here seeks to illuminate potential pathways for implementing a \gls{conrag} system, as depicted in Figure \ref{fig:convqa_system_architecture}, tailored to the use case described in Section \ref{sec:overview}.

The extraction pipeline can be visualized as a tree, where following different paths signifies making decisions with corresponding implications for subsequent steps and components. In Figure \ref{fig:all_components_conrag}, each column represents a decision to be made, although in some cases, choosing not to decide is itself a decision. Dotted lines encircling multiple frames indicate that a combination or ensemble approach is possible. For a better understanding of how to apply this framework to create a potential system implementation, refer to the example in Figure \ref{fig:example_decission_tree}. As previously mentioned, the framework does not prescribe specific models (e.g., BERT, PaLM, etc.) but rather conceptual approaches (e.g., Cross-Encoder). The example in Figure \ref{fig:example_decission_tree} represents a simple zero-shot baseline, which will also be implemented and tested in this thesis Chapter \ref{chap:eval}.

\begin{figure}
    \centering
    \includegraphics[width=\textwidth]{Grafiken/all_components_conrag.png}
    \caption{Overview of all Modules of the Framework}
    \label{fig:all_components_conrag}
\end{figure}

\begin{figure}
    \centering
    \includegraphics[width=\textwidth]{Grafiken/example_decission_tree.png}
    \caption{Example of applying the Framework to a System Implementation}
    \label{fig:example_decission_tree}
\end{figure}



\subsection{extract}
\label{subsec:extract}

\subsection{Retriever}
\label{subsec:retriever}

\subsection{Reader}
\label{subsec:reader}

\subsection{Contextual Query Understanding}
\label{subsec:cqu}


% \section{Question Answering over PDFs}
% \label{sec:qa-over-pdfs}

% \begin{figure}
%     \centering
%     \includegraphics[width=0.8\textwidth]{Grafiken/Possible_Systems.png}
%     \caption{Possible Combinations of Modules to Create an Adapted QA-System}
%     \label{fig:qa-system-combinations}
% \end{figure}

% The possible combinations of modules to create an adapted \gls{qa}-System can be seen in Figure \ref{fig:qa-system-combinations}.


% \subsection{Extract}
% \label{subsec:extract}

% \textbf{Information Extraction:} When it comes to extracting text from PDFs, you have two approaches to choose from: the \textit{visual} approach and the \textit{direct} approach, each with its own advantages. In the \textit{visual} approach, you can create a model for document layout analysis combined with an OCR tool. Ideally, this approach yields a collection, denoted as $M$, comprising paragraphs $m$. These paragraphs vary in length but represent self-contained semantic units from the underlying PDF. On the other hand, the \textit{direct} approach is suitable only for digitally-born PDFs. For such PDFs, you can employ a tool like Py2PDF \cite{noauthor_welcome_nodate}. This results in a corpus $M$ that requires processing to achieve the desired granularity.

% \noindent\textbf{Indexing:} The implementation of indexing depends on the nature of $M$ and the desired granularity of the output. In general, there are three approaches to constructing passages $p$ in the index $P$:

% \begin{enumerate}
%     \item \textit{Paragraphs:} In this approach, $P$ comprises passages $p_i$ such as $P = \{p_1, p_2, \ldots, p_n\}$, where each $p_i$ is derived from $M$ and represents semantic paragraphs from the original document. The length $l$ of each $p_i$ is not fixed, and the number of paragraphs $|P|$ can also vary.
%     \item \textit{Snippets:} When you have a fixed passage length $l$, the concatenated text corpus $M$ is divided into $|M|\bmod{l} + 1$ passages $p$. Alternative approaches may involve specifying minimum and maximum lengths, denoted as $l_{\text{min}}$ and $l_{\text{max}}$. The exact point of division depends on whether a sentence ends within the specified window or not. If a sentence ending is found within the window, the snippet concludes at that point; otherwise, it concludes at the end of the window.
%     \item \textit{Sliding Windows:} This approach utilizes a window size $l$, a concatenated text corpus $M$, and a step size $s$. The window slides over the text corpus $M$, and the text within the window is used as a passage $p$. This results in $\frac{|M| - l}{s}$ passages, denoted as $P = \{p_1, p_2, \ldots, p_n\}$.
% \end{enumerate}

% \textit{Paragraphs} may seem like the most intuitive choice, but the indefinite length of passages, denoted as $p$, can be an issue. \textit{Snippets} have the advantage of having almost uniform lengths. However, a downside could be that important connections between sentences are lost, potentially leading to the loss of information. \textit{Sliding Windows} offer the advantage of uniform lengths and the ability to capture connections between sentences. However, the downside is the high number of passages, denoted as $p$, that need to be indexed, along with potential issues related to data cleanliness.

% \noindent\textbf{Synthetic Training Data Generation:} In this pipeline, the prompt-based \gls{qg} method known as PROMPTAGATOR \cite{dai_promptagator_2022} is used to generate a synthetic \gls{qa} dataset for ColBERTv2 \cite{santhanam_colbertv2_2022} based on $P$. The goal is to create triples of $(q, p^{+}, p^{-})$, similar to those found in datasets like MS MACRO \cite{bajaj_ms_2018}. For the task of $E_{qg}(p) := q$, a \gls{s2s} model, specifically a \gls{llm}, will be employed. This technique can be interpreted as knowledge distilation from the LLM to the later trained retriever. Similar to PROMPTAGATOR, there are two approaches to consider:

% \begin{enumerate}
%     \item \textit{Zero-Shot:} In this approach, a single prompt is executed to generate a question $q_i$ corresponding to a passage $p_i$, all without the need for any supervised dataset.
    
%     \item \textit{Few-Shot:} This approach uses $k$ supervised pairs $(q_j, p^{+}_{j})^{k}$ to generate a question $q_i$ corresponding to a passage $p_i \in P$.
% \end{enumerate}

% The prompt used for \textbf{zero-shot \gls{qg}}, where $p_i$ represents a passage from $P$, is:

% \verb|f'{p_i} Read the passage and generate a corresponding query.'|

% The prompt used for \textbf{few-shot \gls{qg}}, where $q_j$ represents a question, $p_j$ the corresponding passage, and $p$ the passage for which a question is being generated, is:

% \verb|f'Passage: {p_1} Question: {q_1} XXX Passage: {p_2} Question: {q_2}|

% \verb|XXX ... XXX Passage: {p} Question:'|

% The result of either of these approaches will be a synthetic training dataset of $(q_s, p^{+})$ tuples. This dataset can be used for fine-tuning the retriever.

% A major issue associated with this form of \textbf{\gls{qg}} is its strict limitation to extractive questions, where the evidence can be derived from a single passage only. This limitation significantly constrains this pipeline. However, it does not necessarily prevent the system from answering more complex questions, such as multi-hop questions. These can be addressed by the \textit{retriever} and \textit{reader} modules.

% %%%%%

% \subsection{Retrieve}
% \label{subsec:retrieve}

% \textbf{Out-of-Domain Retrievers:} The easiest implementation is the out-of-domain usage of retrievers without fine-tuning and the need for generating a training dataset. Three major retrievers seem promising due to their performance on the BEIR \cite{thakur_beir_2021} out-of-domain benchmark for retrievers:

% \begin{enumerate}
%     \item \textit{BM25} is the standard Sparse Retriever based on lexical probabilistic matching between the query $q$ and passages $p$.
%     \item \textit{Large DPRs} are Dense Retrievers based on large encoders. They utilize typical dense retrieval paradigms and are a primary approach in open-source projects like \textit{Langchain} \cite{noauthor_langchain-ailangchain_nodate}.
%     \item \textit{LaPraDoR} is a hybrid retriever based on a broadly trained Representation-based Retriever, similar to (2), combined with lexical weighting (1).
% \end{enumerate}

% The \gls{laprador} utilizes the advantages of both lexical and semantic search. Given a question $q$ and a passage $p$, the semantic similarity $\text{sim}(q,d)$ is calculated using a \gls{dpr} model. In addition, the lexical similarity $\text{BM25}(q,d)$ is calculated. The final score $\text{score}(q,d)$ is computed as follows:

% \begin{equation}
%     \mathbf{score}(q, d) = \mathbf{sim}(q, d) \cdot \mathbf{BM25}(q, d)
% \end{equation}

% This approach achieves state-of-the-art performance on the BEIR benchmark without the need for fine-tuning. It serves as the ideal off-the-shelf component for the desired \gls{qa}-System.

% \noindent\textbf{Fine-Tuning Retrievers:} Fine-tuning is a challenging task in the absence of a supervised dataset, especially when dealing with a Representation-Interaction Retriever like ColBERTv2. Currently, there is no clear reference on how to fine-tune a Representation-Interaction Retriever like ColBERTv2 on synthetic data. To address this gap, this thesis proposes the approach depicted in Figure \ref{fig:retriever-fine-tuning}, which combines elements from the training processes of PROMPTAGATOR \cite{dai_promptagator_2022}, the original \gls{dpr} \cite{karpukhin_dense_2020}, and ColBERTv2 \cite{santhanam_colbertv2_2022}.

% \begin{figure}
%    \centering
%     \includegraphics[width=0.8\textwidth]{Grafiken/Training.png}
%     \caption{Fine-Tuning Process for Retriever}
%     \label{fig:retriever-fine-tuning} 
% \end{figure}

% A crucial aspect of this fine-tuning approach is the utilization of an already well-performing out-of-domain retriever as a baseline. This baseline retriever can distill its knowledge into the retriever undergoing training. For example, \gls{dpr} used BM25, while ColBERTv2 employed MiniLM \cite{wang_multi-passage_2019}, a 22M-parameter Interaction-based retriever. A useful guideline for selecting the model is to consult the BEIR leaderboard.

% In the first step, the \gls{ood} model must retrieve the top $k$ passages, denoted as $p_i$, for each synthetic $(q_s,p^{+})$ pair. To generate numerous high-quality negative triples, denoted as $(q_s, p^{+}, p^{-})$, for every retrieved passage $p_i$ (where $p_i \neq p^{+}$), the triple $(q_s, p^{+}, p_i)$ is added to the training dataset.

% In the second step, the target retriever is trained for $s$ iterations. The loss function employed is the negative log likelihood, as defined in Section \ref{subsec:qa_retrieval}. During training, in-batch negatives are utilized. Let $Q$ and $P$ represent the $(B \times d)$ matrices of question and passage embeddings in a batch of size $B$. The matrix $S = Q P^{T}$ contains rows where each corresponds to a question paired with all other passages in the batch. The passages from all the other data points act as negatives for the question $q$.

% In the third step, the synthetic dataset is subjected to filtering. PROMPTAGATOR demonstrated promising results of filtering data by a network trained on the data. For this filtering process, retrieval is performed using both the newly trained model and the \gls{ood} model for a question $q_s$. If neither model retrieves the corresponding passage $p^{+}$ for the synthetic question within their top $k$, that question is removed from the dataset.

% Steps two and three are repeated once during fine-tuning.
% \subsection{Read}
% \label{subsec:read}

% \textbf{Out-of-Domain Readers:} There exists no benchmark for the application of zero-shot or \gls{ood} Readers. As Pereira et. al. \cite{pereira_visconde_2022} point out in their experimental results, the zero-shot performance of \gls{llm}s as Generative Readers is state-of-the-art and thus needs no fine-tuning and can even perform in a zero-shot setting. Luo et. al. \cite{luo_choose_2022} also pointed out, that the \gls{ood} performance of Extractive Readers is higher than the of Generative ones, when it comes to \gls{prlm}. Threfore a good \gls{ood} model choice is UnifiedQA-v2 \cite{khashabi_unifiedqa-v2_2022}, which is based on T5 \cite{raffel_exploring_2023} and was trained on 20 diverse datasets.

% \noindent\textbf{Fine-Tuning Readers:} Extractive Readers depend on datasets of the form $(q, c, a_span)$, whereas $q$ is a question, $c$ the context and $a_span$ an indication of which tokens of $c$ correspond to the desired answer. Similar for Generative Readers, which need $(q, c, a)$ datasets, whreas $a$ is just a text based answer to question $q$. The training process for the reader is easier and more straightforward as for the retriever. Given the already filtered synthetic trainings dataset, this can be used for training of the reader.

% \subsection{Orchestration}
% \label{subsec:qa_orchestration}

% Similar to \gls{r2d2}\cite{fajcik_r2-d2_2021} and LaPraDoR \cite{xu_laprador_2022} and others, an orchestration of retrievers and readers will be adopted in order to achieve highest \gls{ood} performance. This involves the following steps:

% \begin{enumerate}
%     \item \textit{Retrieval:} The $k$-top identified passages $P = \{p_1, p_2, \ldots, p_k\}$ are retrieved for a given question $q$ receive a similarity indication $\text{sim}(q,p_k)$ by the retriever. Next to this score, the BM25 score is calculated $BM25(q,p_k)$. The final score is calculated as the weighting of the similarity by BM25:     $score(q, d) = sim(q, d) \cdot BM25(q, d)$
%     \item \textit{Reader:} As for the Reader two readers will be executed at the same time: An extractive and one generative reader. Both 

% \end{enumerate}


% \section{Conversational Question Answering System}
% \label{sec:conv-qa-system}

%%%%%%%%%%%%%%%%%%%%%%%%%%%%%%%%%%%%%%%%%%%%%%%%%%%%%%%%%%%%
\newpage
%%%%%%%%%%%%%%%%%%%%%%%%%%%%%%%%%%%%%%%%%%%%%%%%%%%%%%%%%%%%

\chapter{Experimental Evaluation}
\label{chap:eval}

This chapter is the evaluation of the proposed solution. It consists of laying out different possible solutions to the given problem.

%%%%%%%%%%%%%%%%%%%%%%%%%%%%%%%%%%%%%%%%%%%%%%%%%%%%%%%%%%%%%
\newpage
%%%%%%%%%%%%%%%%%%%%%%%%%%%%%%%%%%%%%%%%%%%%%%%%%%%%%%%%%%%%

\chapter{Conclusions and Future Work}
\label{chap:concl}

This thesis serves as an introduction guide for developing conversational question-answering systems tailored to specific use cases. Beginning with an extensive overview of the field in Section \ref{chap:grundlagen}, we delve into the essential algorithms and concepts necessary for constructing such systems.

Focusing on practical implementations, Section \ref{chap:main} narrows down the implementation landscape by focusing on conversational retrieval-augmented generators for question-answering. We outline a system comprising four distinct components: Extract, CQU, Retriever, and Reader. Each component addresses its own well-defined problem domain, shedding light on the major challenges associated with them. Notable contributions include the detailed breakdown of the often-overlooked Extract component into various operations and the identification of challenges faced by the Retriever component, both with the knowledge base and the evidence set for the Reader. Additionally, we dissect the tasks of the Reader component into smaller challenges and propose multiple existing and novel solutions to tackle them. Ultimately, the main contribution lies in providing a decision map for constructing a conversational question-answering system, based on the diverse components and implementation possibilities.

To validate the feasibility of the decision map, Section \ref{chap:eval} presents the construction and evaluation of a default system primarily based on zero-shot components. Various evaluation approaches are introduced, implemented, and tested within this framework. Notably, the component-wise evaluation using synthetic data yields insights into the feasibility of synthetic datasets for metric-based component evaluation. Challenges such as complex or abstract questions pose difficulties in using synthetic datasets containing tuples of question-answer pairs and evidence, particularly applicable for factoid questions. Furthermore, the performance of two smaller \gls{llm}s, \textit{leo-hessian-7B-chat} and \textit{Llama2-7B-chat}, is compared against \textit{gpt-3.5-turbo}. Performance issues in applications of the smaller \gls{llm}s are identified, necessitating considerations when utilizing them in conversational question-answering systems. These issues primarily revolve around consistency in text generation quality and the suitability of the \gls{llm}s for the role of a Reader in the conversational setting, attributed to their inferior capabilities compared to larger \gls{llm}s in identifying pertinent information from the evidence set.

In Section \ref{subsec:data-augmentation-quality}, we discuss the identified limitations of the chosen approaches, which is necessary reading for anyone interested in building their own conversational question-answering system and how to evaluate it.

\vspace{\baselineskip}
\noindent We recommend any reader interested in building their own conversational question-answering system to start with this thesis as a foundational guide. They can then proceed to build a basic system similar to Section \ref{sec:setup} and evaluate it in an end-to-end manner as described in Section \ref{subsec:rag-eval}. This approach provides the quickest way to understand the specific challenges of their use case and how to further iterate and improve their system.

\vspace{\baselineskip}
\noindent Regarding future work, we propose the following directions for research:
\begin{itemize}
    \item \textbf{Practical Feasibility of Conversational Question Answering Systems:} Evaluate the costs and user experience associated with implementing conversational question-answering systems, particularly considering the significant expenses linked with utilizing \gls{llm}s. This evaluation should delve into whether it is financially viable to construct a search system relying on \gls{llm}s, especially given the escalating costs associated with the growing complexity of utilized approaches and the increasing number of \gls{llm} inference runs required. 

    \item \textbf{Constrained System Design:} Develop a search system that strikes a balance between resource utilization and search result quality. This endeavor may entail amalgamating multiple system components, implementing post-processing techniques on the evidence set, and orchestrating a series of reader components to create a system capable of addressing diverse question types while maintaining efficient resource usage. Relying solely on \gls{llm}-based approaches, such as \gls{rag}, may not always be the optimal solution for all use cases and setups. It might be more resource-efficient to employ a combination of smaller \gls{llm}s for different subtasks, or to explore techniques like quantization and incorporating extractive readers as post-processing steps, among others.

    \item \textbf{Improving the Retriever Component:} Enhance the performance of the retriever component, specifically focusing on its ability to locate all relevant contexts within the evidence set. This endeavor could entail investigating novel methodologies, such as employing agent-like retrievers or capitalizing on the structural elements of documents to direct the extraction and retrieval process. E.g. aligning the retrieval process with the search strategies employed by humans, who often hierarchically navigate through documents to pinpoint relevant passages, may offer valuable insights for enhancing retriever performance.

    \item \textbf{PDF Data Extraction:} Develop a universal solution for extracting textual or multimodal information from PDFs. This includes addressing the challenges of existing extraction operations, which yield different document models and may produce inconsistent passages. 
\end{itemize}

% Your previous chapters

% Start of the appendix
\appendix

\chapter{Appendix A}
\label{app:appendixA}
\section{Implementation Trees}
\label{ref:appendixA-implementation}

\subsection{Example}

\begin{figure}
    \centering
    \includegraphics[width=\textwidth]{Grafiken/example_decission_tree.png}
    \caption{Example Implementation Zero-Shot Baseline}
    \label{fig:example-implementation-tree}
\end{figure}

\subsection{Extract}
 % Include the content of your appendix file

% References (Literaturverzeichnis):
% a) Style (with abbreviations: use alpha):
% see
% https://de.wikibooks.org/wiki/LaTeX-W%C3%B6rterbuch:_bibliographystyle
% for the different formats and styles

% \bibliographystyle{apalike}
\bibliographystyle{apalike}
% b) The File:
\bibliography{references}

\end{document}